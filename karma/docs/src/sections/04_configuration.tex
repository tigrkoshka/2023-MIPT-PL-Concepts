\hypertarget{config}{
	\section{Execution configuration}
}

A \St{Karma} computer as a whole as well as any specific execution
has a configuration.

The configuration allows to:

\begin{itemize}
	\item block the access to some registers
	\item block the access to the code segment of the memory
	\item block the access to the constants segment of the memory
	\item \textit{bound} the stack, i.e.\ specify the maximum stack size,
	so that if the stack grows beyond it, a \textit{stack overflow} error occurs
\end{itemize}

The access to both the registers and the code and constants segments of
the memory can be blocked either for writing (\textit{write block}) or for both
reading and writing (\textit{read-write block}).
The former raises an execution error in case a program tries to
\textit{manually} modify the contents of the blocked register/data segment,
but allows for the program to read the values from the register/data segment.
The latter raises an execution error in case a program tries to
\textit{manually} either modify or read the contents of the blocked
register/data segment.

Note that regardless of whether such blocks are put in place,
the \St{r13}--\St{r15} registers will still be available for
\textit{internal usage}, that is, via the \hyperlink{cmd:flags}{jumps},
the \hyperlink{cmd:stack}{stack-related} and
the \hyperlink{functions:commands}{function call} commands.
The \St{r15} register will also be available for the \St{Karma} computer
internal machinery to maintain the program execution
(see the \hyperlink{r15}{\St{r15} register} section for details).

The configuration for a specific execution is the \textit{strictest} combination
of its own configuration and the common \St{Karma} computer configuration.
That is, if a register or a memory segment is specified to be blocked in either
one of them, it is blocked for the execution.
Similarly, if the stack is specified to be bounded in either one of them,
it is bounded for the execution.
If it is specified to be bounded by both configurations,
its resulting maximum size is the minimal one of the specified sizes.

As stated \hyperlink{standard:notes}{above}, by default, no registers or
memory segments are blocked, and the stack is unbounded.