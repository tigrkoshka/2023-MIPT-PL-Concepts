\subsubsection{Bitwise operators}

\cmdtable{Bitwise operators}{
    10 & \St{not} & \Ss{RI} & \RIcmd{not}{r1}{0} &
    \St{$\text{r1} = \text{\textasciitilde r1}$} \\

    \hline

    11 & \St{shl} & \Ss{RR} & \RRcmd{shl}{r1}{r2}{1} &
    \St{$\text{r1} \mathrel{\ll}= \text{r2} + \text{1}$} \\

    \hline

    12 & \St{shli} & \Ss{RI} & \RIcmd{shli}{r1}{2} &
    \St{$\text{r1} \mathrel{\ll}= \text{2}$} \\

    \hline

    13 & \St{shr} & \Ss{RR} & \RRcmd{shr}{r1}{r2}{-4} &
    \St{$\text{r1} \mathrel{\gg}= \text{r2} - \text{4}$} \\

    \hline

    14 & \St{shri} & \Ss{RI} & \RIcmd{shri}{r1}{2} &
    \St{$\text{r1} \mathrel{\gg}= \text{2}$} \\

    \hline

    15 & \St{and} & \Ss{RR} & \RRcmd{and}{r4}{r6}{3} &
    \St{$\text{r4} \mathrel{\&}= \text{r6} + \text{3}$} \\

    \hline

    16 & \St{andi} & \Ss{RI} & \RIcmd{andi}{r5}{2} &
    \St{$\text{r5} \mathrel{\&}= \text{2}$} \\

    \hline

    17 & \St{or} & \Ss{RR} & \RRcmd{or}{r3}{r2}{-2} &
    \St{$\text{r3} \mathrel{|}= \text{r2} - \text{2}$} \\

    \hline

    18 & \St{ori} & \Ss{RI} & \RIcmd{ori}{r6}{100} &
    \St{$\text{r6} \mathrel{|}= \text{100}$} \\

    \hline

    19 & \St{xor} & \Ss{RR} & \RRcmd{xor}{r1}{r5}{0} &
    \St{$\text{r1} \mathrel{^\wedge}= \text{r5}$} \\

    \hline

    20 & \St{xori} & \Ss{RI} & \RIcmd{xori}{r1}{127} &
    \St{$\text{r1} \mathrel{^\wedge}= \text{127}$} \\
}
\paragraph{\St{not}}\

The immediate value is ignored, but per our agreement must be present for simplicity.

\vspace{-0.35cm}
\paragraph{Other bitwise operators}\

All other bitwise operators take two \St{uint32} values obtained from
the specified operands in the \hyperlink{types:twos_complement}{common manner}
and produce a \St{uint32} value as a result.
The right hand side operand of the bitwise shift operators must be less than
the size of a machine word (i.e.\ no more than 31), otherwise an execution
error occurs.
