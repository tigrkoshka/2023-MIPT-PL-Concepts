\hypertarget{cmd:data_transfer}{
    \subsubsection{Data transfer commands}
}

\vspace{-0.1cm}

\cmdtable{Data transfer commands} {

    39 & \St{lc} & \Ss{RI} & \RIcmd{lc}{r7}{123} &
    \St{$\text{r7} = \text{123}$} \\

    \hline

	40 & \St{la} & \Ss{RM} & \RIcmd{la}{r5}{0xab1f} &
    \St{$\text{r5} = \text{0xab1f}$} \\

    \hline

    41 & \hypertarget{cmd:mov}{\St{mov}} & \Ss{RR} & \RRcmd{mov}{r0}{r3}{10} &
    \St{$\text{r0} = \text{r3} + \text{10}$} \\

    \hline

    42 & \St{load} & \Ss{RM} & \RMcmd{load}{r3}{0xffc2} &
    \St{$\text{r3} = {}^*(\text{0xffc2})$} \\

    \hline

    43 & \St{load2} & \Ss{RM} & \RMcmd{load2}{r4}{13224} &
    \cmdcellalign{
        \text{r4} = {}^*(\text{13224}) \\
        \text{r5} = {}^*(\text{13225}) \\
    } \\

    \hline

    44 & \St{store} & \Ss{RM} & \RMcmd{store}{r1}{057123} &
    \St{${}^*(\text{057123}) = \text{r1}$} \\

    \hline

    45 & \St{store2} & \Ss{RM} & \RMcmd{store2}{r0}{0X15fc} &
    \cmdcellalign{
            {}^*(\text{0X15fc}) = \text{r0} \\
        {}^*(\text{0X15fd}) = \text{r1} \\
    } \\

    \hline

    46 & \St{loadr} & \Ss{RR} & \LongRRcmd{loadr}{r8}{r1}{15} &
    \St{$\text{r8} = {}^*(\text{r1} + \text{15})$} \\

    \hline

    47 & \St{loadr2} & \Ss{RR} & \LongRRcmd{loadr2}{r2}{r0}{12} &
    \cmdcellalign{
        \text{r2} = {}^*(\text{r0} + \text{12}) \\
        \text{r3} = {}^*(\text{r0} + \text{13}) \\
    } \\

    \hline

    48 & \St{storer} & \Ss{RR} & \LongRRcmd{storer}{r2}{r3}{3} &
    \St{${}^*(\text{r3} + \text{3}) = \text{r2}$} \\

    \hline

    49 & \St{storer2} & \Ss{RR} & \LongRRcmd{storer2}{r5}{r2}{13} &
    \cmdcellalign{
            {}^*(\text{r2} + \text{13}) = \text{r5} \\
        {}^*(\text{r3} + \text{14}) = \text{r6} \\
    } \\
}

\vspace{-0.2cm}

\paragraph{\St{lc}, \St{la}, \St{mov}}\

These commands store the \St{uint32} value obtained from the operands in
the {\hyperlink{types:twos_complement}{common manner} to the specified receiver
register. The \St{la} command is similar to \St{lc}, but allows for labels usage (see \hyperlink{labels}{Labels} section for details).

\vspace{-0.35cm}

\paragraph{Register-memory data transfer}\

All the other commands are used to transfer data between the registers and
the memory.

Those of them which have the \Ss{RR} format treat the \St{uint32} value obtained
from the operands in the {\hyperlink{types:twos_complement}{common manner}
as a memory address.
If this value does not represent a valid address, i.e.\ is greater or equal
to $2^{20}$, an execution error occurs.
Note that the \St{storer} and the \St{storer2} commands treat the receiver
register operand as the opposite, i.e.\ as the \textit{source} for a value
to be stored into the memory.
These are the only two \Ss{RR} commands with such behaviour.

For the \Ss{RM} format commands the memory cell is specified by the 20-bit
\textit{unsigned} memory address operand.
Since any 20-bit unsigned value represents a valid memory address, an execution
error at this stage cannot occur in this case.

The commands with a `\St{2}' suffix perform the same action as the respective
commands without the suffix twice: once with the register and the memory cell
specified by the operands and then with the next register and the next memory cell.
Therefore, the specified register cannot be \St{r15} and the specified memory
cell cannot be \St{0xfffff} (there are no next ones for them).
If such a situation does occur, it results in an execution error.
