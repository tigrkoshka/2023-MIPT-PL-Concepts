\subsubsection{System commands}

\cmdtable{System commands}{
    0 & \St{halt} & \Ss{RI} &

    \St{halt r1 0} & Stop processor \\

    \hline

    1 & \St{syscall} & \Ss{RI} &

    \St{syscall r0 100} & System call \\
}
\paragraph{\St{halt}}\

The \St{halt} command sends an interruption signal to the CPU, which halts it
until the next external signal is received, after which the execution is continued.
The immediate value operand is ignored.

\vspace{-0.35cm}
\paragraph{\St{syscall}}\

The \St{syscall} command's immediate operand specifies the call code.
The semantics of those codes is described in Table 4:

    {
    \vspace{-0.4cm}
    \renewcommand{\arraystretch}{1.4}
    \begin{table}[h!]
        \centering
        \caption{System call codes}
        \vspace{2mm}
        \begin{tabular}{|
                >{\centering\arraybackslash} m{1.0cm} |
                >{\centering\arraybackslash} m{2.3cm} |
                >{}                          m{8cm}   |
                >{}                          m{3cm}   |
        }
            \hline
            Code & Name             & Description                                           & Register operand  \\
            \hline
            0    & \St{EXIT}        & Finish execution with the code from the register      & Source            \\
            100  & \St{SCANINT}     & Get a \St{uint32} from \St{stdin}                     & Receiver          \\
            101  & \St{SCANDOUBLE}  & Get a \St{double} value from \St{stdin}               & Low bits receiver \\
            102  & \St{PRINTINT}    & Output a \St{uint32} to \St{stdout}                   & Source            \\
            103  & \St{PRINTDOUBLE} & Output \St{double} value to \St{stdout}               & Low bits source   \\
            104  & \St{GETCHAR}     & Get a single \St{ASCII} character from \St{stdin}     & Receiver          \\
            105  & \St{PUTCHAR}     & Output a single \St{ASCII} character to \St{stdout} & Source            \\
            \hline
        \end{tabular}
    \end{table}
}

Specifying a value for the \St{syscall} immediate operand not listed in
the table above leads to an execution error.

If the source register for the \St{PUTCHAR} system call contains a value greater
than 255 (that is, not representing an \St{ASCII} character), an execution error
occurs.
