\subsubsection{Real-valued operators}

\cmdtable{Real-valued operators}{
    21 & \St{itod} & \Ss{RR} & \RRcmd{itod}{r2}{r5}{5} &
    \St{$(\text{r3}, \text{r2}) = \text{double}(\text{r5} + \text{5})$}
    \\

    \hline

    22 & \St{dtoi} & \Ss{RR} & \RRcmd{dtoi}{r2}{r5}{0} &
    \St{$\text{r2} = \text{uint32}((\text{r6}, \text{r5}))$}\\

    \hline


    23 & \St{addd} & \Ss{RR} & \RRcmd{addd}{r2}{r5}{0} &
    \St{$(\text{r3}, \text{r2}) \mathrel{+}= (\text{r6}, \text{r5})$} \\

    \hline

    24 & \St{subd} & \Ss{RR} & \RRcmd{subd}{r1}{r6}{0} &
    \St{$(\text{r2}, \text{r1}) \mathrel{-}= (\text{r7}, \text{r6})$} \\

    \hline

    25 & \St{muld} & \Ss{RR} & \RRcmd{muld}{r0}{r2}{0} &
    \St{$(\text{r1}, \text{r0}) \mathrel{\ast}= (\text{r3}, \text{r2})$} \\

    \hline

    26 & \St{divd} & \Ss{RR} & \RRcmd{divd}{r1}{r3}{0} &
    \St{$(\text{r2}, \text{r1}) \mathrel{/}= (\text{r4}, \text{r3})$} \\
}

\vspace{0.5cm}

For the \St{double} values storage please refer to
\hyperlink{types:two_words_storage}{this section}.
For the double values memory representation please refer to
\href{https://en.wikipedia.org/wiki/Floating-point_arithmetic}{Wiki},
specifically to the
\href{https://en.wikipedia.org/wiki/Floating-point_arithmetic#Floating-point_numbers}
{Overview} and
\href{https://en.wikipedia.org/wiki/Floating-point_arithmetic#Internal_representation}
{Internal representation} sections.

\vspace{-0.35cm}
\paragraph{\St{itod}}\

The source modifier is added to the source register in
the \hyperlink{types:twos_complement}{common manner}, after which the resulting
\St{uint32} value is converted to \St{double} and stored into the two registers
starting from the specified receiver register.

\vspace{-0.35cm}
\paragraph{\St{dtoi}}\

The source modifier is ignored, but per our agreement must be present for
simplicity.

The \St{double} value obtained from the two registers starting from the source
register is converted to \St{uint32} by discarding the fractional part
and stored into the receiver register.
If the resulting value does not fit into a register, an execution error occurs.

\vspace{-0.35cm}
\paragraph{\St{Real-valued arithmetic operators}}\

The source modifier is ignored for all real-valued arithmetic operators,
but per our agreement must be present for simplicity.
