\hypertarget{cmd:flags}{
    \subsubsection{Comparisons and jumps}
}

\cmdtable{Comparisons and jumps} {
    27 & \St{cmp} & \Ss{RR} & \RRcmd{cmp}{r0}{r1}{2} &
    \St{$\text{r0} <=> \text{r1} + \text{2}$} \\

    \hline

    28 & \St{cmpi} & \Ss{RI} & \RIcmd{cmpi}{r0}{0} &
    \St{$\text{r0} <=> \text{0}$}  \\

    \hline

    29 & \St{cmpd} & \Ss{RR} & \RRcmd{cmpd}{r1}{r4}{0} &
    \St{$(\text{r2}, \text{r1}) <=> (\text{r5}, \text{r4})$} \\

    \hline

    30 & \St{jmp} & \Ss{J} & \Jcmd{jmp}{0xffa} & Unconditional jump \\

    \hline

    31 & \St{jne} & \Ss{J} & \Jcmd{jne}{0xd471} & Jump if not equal \\

    \hline

    32 & \St{jeq} & \Ss{J} & \Jcmd{jeq}{05637} & Jump if equal \\

    \hline

    33 & \St{jle} & \Ss{J} & \Jcmd{jle}{1234} & Jump if less or equal \\

    \hline

    34 & \St{jl} & \Ss{J} & \Jcmd{jl}{0X1e4b} & Jump if less \\

    \hline

    35 & \St{jge} & \Ss{J} & \Jcmd{jge}{29834} & Jump if greater or equal \\

    \hline

    36 & \St{jg} & \Ss{J} & \Jcmd{jg}{03457} & Jump if greater \\
}

\paragraph{Flags register}\

To allow for basic execution branching, an additional \St{flags} register
is supported.
It holds the result of the latest comparison.
Only the lowest 6 bits of this register are used.
The semantics of those bits is described in \hyperlink{flags:bits}{Table 11}.

\hypertarget{flags:bits}{}
{
    \vspace{-0.4cm}
    \renewcommand{\arraystretch}{1.4}
    \begin{table}[h!]
        \centering
        \caption{\St{flags} register bits semantics (low-to-high, 0-indexed)}
        \vspace{2mm}
        \begin{tabular}{| c | c |}
            \hline
            0 & Equal            \\
            1 & Not equal        \\
            2 & Greater          \\
            3 & Less             \\
            4 & Greater or equal \\
            5 & Less or equal    \\
            \hline
        \end{tabular}
    \end{table}
}

Several bits may be simultaneously filled.
For example, if the latest comparison resulted in equality,
the value of the \St{flags} register will be $110001_2$, because equality
causes the `less/greater or equal' conditions to also be true.

\vspace{-0.35cm}

\paragraph{\St{cmp}, \St{cmpi}, \St{cmpd}}\

These are comparison operators.
They compare the two values provided by the operands and store the result to
the \St{flags} register according to the \hyperlink{flags:bits}{\St{flags}
register bits semantics} described above.
These are the only commands that store values in the \St{flags} register.

The \St{cmp} and \St{cmpi} command compare the \St{uint32} value from
the receiver register to the \St{uint32} value obtained from the operands in
the {\hyperlink{types:twos_complement}{common manner}.

The \St{cmpd} command ignores the immediate value operand and compares the two
\St{double} values specified by the receiver and the source register operands.
For the \St{double} values storage please refer to
\hyperlink{types:two_words_storage}{this section}.

\vspace{-0.35cm}

\hypertarget{cmd:jump}{}

\paragraph{Jump commands}\

These commands implement execution branching.
All of them, except for the \St{jmp} command, do nothing unless the condition
mentioned in the table above is satisfied, that is, if the respective bit
(according to the \hyperlink{flags:bits}{\St{flags} register bits semantics})
is filled in the \St{flags} register.
The \St{jmp} command is performed unconditionally.

If one of these commands take effect, the next executed instruction will be
the one stored in the provided memory address operand (because the \St{Karma}
computer has a \href{https://en.wikipedia.org/wiki/Von_Neumann_architecture}
{Von Neumann architecture}, the instructions' binaries are stored in
the common address space).

These are the only commands that read the value stored in the \St{flags}
register.
