\hypertarget{cmd:stack}{
    \subsubsection{Stack-related commands}
}

\cmdtable{Stack operators and function calls}{
    37 & \hypertarget{cmd:push}{\St{push}} & \Ss{RI} & \RIcmd{push}{r0}{255} &
    \St{${}^\ast(\text{r14}--) = \text{r0} + \text{255}$} \\

    \hline

    38 & \hypertarget{cmd:pop}{\St{pop}} & \Ss{RI} & \RIcmd{pop}{r3}{3} &
    \St{$\text{r3} = {}^\ast(++ \text{r14}) + \text{3}$} \\

}

\vspace{0.4cm}

These commands work directly with the stack, putting \St{uint32} values into
it (\St{push}) and extracting them afterwards (\St{pop}).

For the \St{push} command, the immediate value operand is added to the value
from the specified source register before it is pushed to the stack.
Similarly, for the \St{pop} command, the immediate value operand is added to
the value retrieved (popped) from the stack before it is stored in the specified
receiver register.
The additions happen according to the {\hyperlink{types:twos_complement}
{common rules}.

The \St{r14} register is the stack head pointer
(as stated \hyperlink{registers}{here}).
It always points to the memory cell, in which the data will be put on
\St{push} operation, that is, one memory cell `ahead' of the latest pushed value.
The stack grows \textit{backwards}, i.e.\ the next pushed value is stored in
the \textit{previous} memory cell from the latest one.
