\hypertarget{constants}{
    \subsection{Constants}
}

One can define a constant of any \hyperlink{types:basic}{basic type}
as well as of two additional types: \St{char} and \St{string}.

Syntax sample: \St{uint32 123}

For consistent disassembling purposes any constant's memory representation
is preceded with a one-word unique type identifier.

\vspace{-0.35cm}
\paragraph{\St{uint32}, \St{uint64}}\

These values may be represented in all the same
\hyperlink{operand:representation}{ways} as command operands.

The negative values are processed in the \hyperlink{types:twos_complement}
{common manner}, i.e.\ using the two's complement representation for
the bit size of the respective type.
The values of \St{uint32} type constants which are greater or equal
to $2^{32}$, but are less than $2^{64}$, are taken modulo $2^{32}$.
If the specified value does not fit into \St{uint64}, a compile error occurs.

For storage details see
\hyperlink{types:two_words_storage}{the respective section}.

\vspace{-0.35cm}
\paragraph{\St{double}}\

A \St{double} constant may be represented either in a decimal or
a hexadecimal formats.
Both these formats are described with the
\href{https://en.cppreference.com/w/cpp/string/basic_string/stof}{std::stod}
function of the C++ standard.

For storage details see
\hyperlink{types:two_words_storage}{the respective section}.

\vspace{-0.35cm}
\paragraph{\St{char}}\

A \St{char} constant value must be surrounded with single quotes.

There are a number of \textit{escape sequences} representing certain
special \St{ASCII} characters.
One of those is the `\textbackslash \#' escape sequence, used to represent
a single `\#' character (the \textit{hash sign}, \St{ASCII} code 35).
This escape sequence allows to resolve the confusion between a hash sign
at the start of a comment (see \hyperlink{comments}{Comments} section
for details).
The rest of the supported escape sequences are listed in the
\href{https://en.cppreference.com/w/cpp/language/escape}{Simple escape sequences}
table of the C++ standard.

A \St{char} constant value is stored as a \St{uint32}
from 0 to 255 inclusively in \St{ASCII} encoding.

\vspace{-0.35cm}
\paragraph{\St{string}}\

A \St{string} constant value must be surrounded with double quotes.
Such a value represents a sequence of \St{char} values.
In particular, that means that all the same escape sequences are used.

A \St{string} value is represented in memory as the respective
\St{char} values stored in consecutive memory cells followed by
an additional \textit{zero character} (\St{ASCII} code 0).
The zero character indicates the end of a \St{string}.
