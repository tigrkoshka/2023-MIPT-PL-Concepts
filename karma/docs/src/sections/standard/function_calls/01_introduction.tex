\hypertarget{functions:introduction}{
	\subsubsection{Introduction: calling conventions}
}

This section defines the rules by which the execution of one part of the code
(the \textit{caller}) can be delayed until another part (the \textit{callee})
is executed.
The callee is often also referred to as a \textit{subprogram} or
a \textit{function}, and the delay process is called a \textit{function call}.
The process of passing the execution from the callee back to the caller
is often referred to as the \textit{return from the function}.

For the functions to be meaningful they need to accept \textit{arguments},
i.e.\ values passed from the caller to the callee for the latter to perform
computations dependant on these values.
The arguments of a function are often
also referred to as \textit{parameters}.

The functions often need to calculate some values for the caller to use later.
In such cases these values are called the \textit{return parameters}
of the function and need to be somehow passed to the caller.

The function call procedure inevitably leads to a number of questions that
need to be resolved, such as:

\begin{itemize}
    \item Where the arguments are placed? \\
    Options include in the registers, on the stack, a mix of both,
    or in another memory structures

    \item In what order are the arguments passed? \\
    Options include first-to-last, last-to-first, or something more complex

    \item Whether and how metadata describing the arguments is passed? \\
    The metadata may include the number of arguments, the description
    of their position, or some other technical information

    \item How the functions that take a variable number of arguments
    (\textit{variadic functions}) are handled? \\
    Options include passing the number of arguments as an argument
    itself placed in an obvious position, placing the variable arguments
    in an array in the memory and passing the pointer to it, or keeping
    track of the number of arguments in a separate register

    \item How return parameters are delivered from the callee to the caller? \\
    Options include in the registers, on the stack, a mix of both,
    or in another memory structures

    \item How the task of setting up for and cleaning up after
    a function call is divided between the caller and the callee? \\
    This question refers to the arguments passed to the caller,
    the caller's \textit{local variables}, i.e.\ the values it places
    in the registers, on the stack and in the memory, and the additional
    values (if any) used internally for the function call itself

    \item Which registers are guaranteed to have the same value
    when the callee returns as they did when the callee was called? \\
    Such registers (if any) are called \textit{saved} or \textit{preserved}
    as opposed to the \textit{volatile} registers, which are the rest of them.
    These registers usually have a utility purpose helping to resolve the other
    questions
\end{itemize}

The set of answers to these questions together with the description
of the function call procedure is called the \textit{calling convention}.
There are a number of viable and widely used calling conventions.
Each of them has its advantages and drawbacks, and the choice of the one most
suitable for a specific case is often based on the computer architecture
and the presumed programming process: whether the programs are written
in the assembler language or some higher level languages to be later
compiled into the assembler or the byte code.

Note that a calling convention is exactly that -- a convention.
Nothing on the architecture or the syntax level prevents one from not following
it during their code development.
However, that is considered a bad practice as it can lead to confusion and
unexpected results for the code user.
Besides, the assembler language often provides
\hyperlink{functions:commands}{special commands} that internally
perform the utility actions defined by the convention.
If one wishes to not follow the convention, all the actions must be performed
manually for every function call.

Almost all calling conventions use the term \textit{stack frame}, which
is a part of the stack related to a specific function.
The function's stack frame usually includes its arguments, the address
of the caller instruction to be executed first after the return from
the callee (the \textit{return address)} and the function's local variables.
It may or may not include additional utility purpose values.

There are two more terms used when describing a calling convention:
the \textit{prologue} and the \textit{epilogue} of a function call. \\
The prologue is a set of internal technical instructions performed
when calling a subprogram before the subprogram execution. \\
The epilogue is a set of such instructions performed
when returning from a function before continuing the caller execution.
