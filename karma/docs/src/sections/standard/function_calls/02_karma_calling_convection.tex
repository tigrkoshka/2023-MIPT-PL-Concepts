\subsubsection{\St{Karma} calling convention}

\vspace{-0.2cm}

\hypertarget{karma:call:basic}{}
By the \St{Karma} calling convention, the arguments are passed to
a function via the stack last-to-first, that is, the semantically first argument
is the last one to be pushed to the stack.
No additional arguments metadata is used.
The variadic functions are handled using an additional utility register
\St{r13} (see \hyperlink{r13}{below} for details).
The return parameters are delivered via the registers, starting from \St{r0}
first-to-last, that is, the first return parameter is placed in \St{r0},
the second one -- in \St{r1}, etc.

The setup for and the cleanup after a function call is mainly performed
in its prologue and epilogue, i.e.\ a programmer does not need to explicitly
write their code.
However, some actions still must be performed explicitly
by the assembler code, that is an inevitable part of any calling convention.

The preserved registers are \St{r13}, \St{r14} and \St{r15}.
The semantics of the \St{r14} (stack pointer) register is described in
the \hyperlink{smd:stack}{Stack-related commands} section, and the usage
of the \St{r13} and \St{r15} registers is described below.

\vspace{-0.5cm}

\hypertarget{r13}{
    \paragraph{\St{r13} register}\
}

The \St{r13} register typically contains the stack pointer (\St{r14}) value
before the local variables of the currently executing subprogram.
During the function calling process for a short time it changes its purpose
and contains the stack pointer value before the arguments of
the callee.
At the start of execution it is initialised with the initial stack pointer
value (defined in the executable file, see
\hyperlink{executable}{the respective section} for details),
which is the stack pointer value before the local variables of the main
(initial) subprogram defined by
the \hyperlink{directives:end}{\St{end} directive}.

\vspace{-0.5cm}

\paragraph{\St{r15} register}\

The \St{r15} register \textit{always} contains the address of the next
instruction to be executed.
When executing the binary code, the \St{Karma} executor consecutively checks
the \St{r15} register and executes the command, whose address is stored there.
If at any point the \St{r15} register is corrupted and does not store
a valid memory address
(i.e.\ if the value stored in the \St{r15} register is greater or equal
to $2^{20}$), an execution error occurs.
