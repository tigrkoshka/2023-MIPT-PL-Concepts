\hypertarget{functions:call}{
    \subsubsection{Function call procedure}
}

\vspace{-0.2cm}

Before calling any function, the assembler code must \textit{explicitly}
do the following;

\begin{itemize}
    \item Push the current value of the \St{r13} register
    (i.e.\ the stack pointer before the caller's local variables)
    to the stack \\
    \St{
        \makebox[0.85cm][l]{push}
        \makebox[0.6cm][l]{r13}
        \makebox[0.45cm][l]{0}
    }

    \item Set the \St{r13} register to the current stack pointer value
    (i.e.\ the value before the function arguments)\\
    \St{
        \makebox[0.60cm][l]{mov}
        \makebox[0.60cm][l]{r13}
        \makebox[0.60cm][l]{r14}
        \makebox[0.20cm][l]{0}
    }

    \item Push the arguments for the function to be called to the stack
    last-to-first as described \hyperlink{karma:call:basic}{above}

\end{itemize}

The prologue of a function call does the following

\begin{itemize}
    \item Pushes the return address (i.e.\ the address of the next
    caller instruction, which is stored in the \St{r15} register)
    to the stack \\
    As if with a \hspace{-0.1cm}\St{
        \makebox[0.85cm][l]{push}
        \makebox[0.60cm][l]{r15}
        \makebox[0.45cm][l]{0}
    }

    \item Pushes the current value of the \St{r13} register (i.e.\ the value
    explicitly stored by the assembler code, which is the stack pointer value
    before the subprogram arguments) to the stack \\
    As if with a \hspace{-0.1cm}\St{
        \makebox[0.85cm][l]{push}
        \makebox[0.60cm][l]{r13}
        \makebox[0.45cm][l]{0}
    }

    \item Sets the \St{r13} register to the current stack pointer value
    (i.e.\ the value before the function local variables) \\
    As if with a \hspace{-0.1cm}\St{
        \makebox[0.60cm][l]{mov}
        \makebox[0.60cm][l]{r13}
        \makebox[0.60cm][l]{r14}
        \makebox[0.20cm][l]{0}
    }
\end{itemize}
