\subsubsection{Notes and other calling conventions}

\vspace{-0.2cm}

The suggested calling convention has two drawbacks.

The main one is that a programmer must explicitly manage the \St{r13} register
before pushing the function's arguments to the stack
(see \hyperlink{functions:call}{above} for details).

Another one is that the \St{r13} register does not have one consistent semantics.
Instead, it has two semantics (see \hyperlink{r13}{above} for details),
one of which is only used during the function call procedure.
This also leads to push its value to the stack twice during the function call
procedure (see \hyperlink{functions:call}{above} for details).

These drawbacks, however, allow for a major advantage of the \St{Karma} calling
convention.
The advantage is that there is no need for the programmer to manually clear
the stack from the function arguments or local variables.
It is done automatically by the epilogue
(see \hyperlink{functions:return}{above} for details).
Neither is there a need for any argument metadata, which optimises the stack
usage.

Other calling conventions may manage the removal of the function arguments
and local variables from the stack in one of the following manners:

\begin{itemize}
    \item Passing the responsibility of removing the function arguments
    to the caller, which must clear the stack after the return from
    the function by explicitly moving the stack pointer.
    In this case the \St{r13} register is usually solely used for automatic
    removal of the local variables.

    \item Interpreting the memory address operand of the \St{ret} command
    as the number of arguments to be removed from the stack.
    Like the previous case, in this one the \St{r13} register is also usually
    solely used for automatic removal of the local variables.
    However, this approach has a problem with variadic functions, because when
    writing a variadic function one does not know the exact number of arguments
    that will be passed to it to pass this number as the operand to the \St{ret}
    command.
    One of the possible solutions of this issue may be to pass the number
    of arguments as the first parameter of all variadic functions and
    to introduce a special value for the \St{ret} operand (e.g.\ \St{-1})
    to be interpreted as an instruction to treat the first argument as
    the number of arguments and increment the stack pointer
    by the respective value.

    \item Passing the responsibility of removing the function local variables
    to the callee, which must clear the stack before the return
    by explicitly moving the stack pointer.
    In this case the \St{r13} register is usually solely used for automatic
    removal of the function arguments.
\end{itemize}

These other calling conventions may sometimes optimise the stack usage,
but are mainly meant for the cases when the programmer writes the code in some
higher level languages to be then compiled to assembler.
In such a case, the compiler can substitute the correct value to increment
the stack pointer by knowing at compile time the number of function arguments
and local variables.
In our case, however, when the code is written directly in assembler, such
an approach is bugprone because a single invalid stack pointer increment
will lead to the stack corruption and unexpected behaviour.
