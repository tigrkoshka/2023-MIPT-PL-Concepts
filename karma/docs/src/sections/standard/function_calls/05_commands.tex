\subsubsection{Function call commands}

\cmdtable{Function call comands}{

    \hline

    50 & \St{call} & \Ss{RR} & \RRcmd{call}{r0}{r5}{2} &
    Call the function located \newline
    at the address \St{$\text{r5} + \text{2}$} \\

    \hline

    51 & \St{calli} & \Ss{J} & \St{calli\kern 0.5em 21913} &
    Call the function located \newline
    at the address \St{21913} \\

    \hline

    52 & \St{ret} & \Ss{J} & \St{ret\kern 0.35em 0} &
    Return from the callee \\

}

\paragraph{\St{call}, \St{calli}}\

These commands perform the transfer of control from the caller to the callee
by executing the function call prologue and storing the address
of the next instruction to be executed (i.e.\ the first instruction of the callee)
to \St{r15}.
The \St{call} command additionally stores a copy of the return address
into the provided receiver register.

For the \St{call} command the address of the next instruction to be executed
is obtained from the specified operands in the
\hyperlink{types:twos_complement}{common manner}.
If this value does not represent a valid address,
i.e.\ is greater or equal to $2^{20}$, an execution error occurs.

For the \St{calli} command the address of the next instruction to be executed
is specified by the 20-bit \textit{unsigned} memory address operand.
Since any 20-bit unsigned value represents a valid memory address, no execution
error can occur in this case.


\vspace{-0.35cm}

\paragraph{\St{ret}}\

This command performs the transfer of control from the callee back to the caller
by executing the function call epilogue, after which the \St{r15} register
contains the return address (see \hyperlink{functions:return}{above} for details).
The memory address operand is ignored.
