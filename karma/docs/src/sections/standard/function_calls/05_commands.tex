\hypertarget{functions:commands}{
    \subsubsection{Function call commands}
}

\cmdtable{Function call comands}{

    \hline

    50 & \hypertarget{functions:cmd:prc}{\St{prc}} & \Ss{J} &
    \St{prc\kern 0.35em 0} & Prepare for a function call \\

    \hline

    50 & \St{call} & \Ss{RR} & \RRcmd{call}{r0}{r5}{2} &
    Call the function located \newline
    at the address \St{$\text{r5} + \text{2}$} \\

    \hline

    51 & \St{calli} & \Ss{J} & \St{calli\kern 0.5em 21913} &
    Call the function located \newline
    at the address \St{21913} \\

    \hline

    52 & \St{ret} & \Ss{J} & \St{ret\kern 0.35em 0} &
    Return from the callee \\

}

\vspace{0.5cm}

These commands implement the
\hypertarget{functions:convention}{\St{Karma} calling convention} and encourage
the convention usage.

They do not provide any functionality that cannot be achieved using other
instructions (see the \hyperlink{functions:call}{Function call procedure} and
the \hyperlink{functions:return}{Return from a function procedure} sections
above for the explicit sets of commands that perform the same actions).

Therefore, as described in the
\hyperlink{functions:introduction}{Introduction: calling conventions}
section, the \St{Karma} calling convention does not \textit{enforce}
anything.
One may choose to follow a different convention in their code, but then
they will be unable to use these commands and must perform all the utility
actions required by their convention manually.

\vspace{-0.3cm}
\paragraph{\St{prc}}\

This command must be \textit{explicitly} executed before pushing the function
arguments to the stack.
It performs the preparatory actions described \hyperlink{functions:prep}{above}.

This is the only additional (to the pushing the function arguments to the stack)
explicit action required for the function call.
The rest of utility actions are performed automatically by
the \St{call}/\St{calli} and the \St{ret} commands.

After this command and before the subsequent \St{call} or \St{calli} command
all values pushed to the stack are considered the function arguments.

The memory address operand is ignored, but per our convention must be present
for simplicity.

\vspace{-0.3cm}
\paragraph{\St{call}, \St{calli}}\

These commands perform the transfer of control from the caller to the callee
by executing the \hyperlink{functions:prologue}{function call prologue}
and storing the address of the next instruction to be executed
(i.e.\ the first instruction of the callee) to \St{r15}.
The \St{call} command additionally stores a copy of the return address
into the provided receiver register.

For the \St{call} command the address of the next instruction to be executed
is obtained from the specified operands in the
\hyperlink{types:twos_complement}{common}
\hyperlink{types:twos_complement}{manner}.
If this value does not represent a valid address,
i.e.\ is greater or equal to $2^{20}$, an execution error occurs.

For the \St{calli} command the address of the next instruction to be executed
is specified by the 20-bit \textit{unsigned} memory address operand.
Since any 20-bit unsigned value represents a valid memory address, no execution
error can occur in this case.

Note that since the \hyperlink{functions:prologue}{function call prologue}
pushes the values from the \St{r15} and the \St{r13} registers to the stack,
after the prologue execution, the arguments start from the third latest value
in the stack.
For example the first argument can be stored to the \St{r0} register
via a \hspace{-0.1cm}\St{
    \makebox[1.0cm][l]{loadr}
    \makebox[0.40cm][l]{r0}
    \makebox[0.60cm][l]{r14}
    \makebox[0.20cm][l]{3}
}
\hspace{-0.2cm} command (note the non-zero source modifier operand).

\vspace{-0.35cm}
\paragraph{\St{ret}}\

This command performs the transfer of control from the callee back to the caller
by executing the \hyperlink{functions:return}{function call epilogue},
after which the \St{r15} register contains the return address.

The memory address operand is ignored, but per our convention must be present
for simplicity.
