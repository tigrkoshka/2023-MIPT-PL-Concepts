\hypertarget{functions:return}{
    \paragraph{Return from a function procedure}\
}

Neither the callee nor the caller need to perform any explicit actions
on a return from a function.
Everything is done automatically by the epilogue.
The epilogue undoes the actions performed during the function call procedure.
Specifically, it does the following:

\begin{itemize}
    \item Sets the stack pointer to the current value of the \St{r13} register
    (i.e.\ removes the local variables of the callee from the stack) \\
    As if with a \hspace{-0.1cm}\St{
        \makebox[0.60cm][l]{mov}
        \makebox[0.60cm][l]{r14}
        \makebox[0.60cm][l]{r13}
        \makebox[0.20cm][l]{0}
    }

    \item Pops a value from the stack to the \St{r13} register
    (that is the value of the stack pointer before the function arguments) \\
    As if with a \hspace{-0.1cm}\St{
        \makebox[0.60cm][l]{pop}
        \makebox[0.60cm][l]{r13}
        \makebox[0.45cm][l]{0}
    }

    \item Pops a value from the stack to the \St{r15} register
    (that is the address of the next caller's instruction to be executed) \\
    As if with a \hspace{-0.1cm}\St{
        \makebox[0.60cm][l]{pop}
        \makebox[0.60cm][l]{r15}
        \makebox[0.45cm][l]{0}
    }

    \item Sets the stack pointer to the current value of the \St{r13} register
    (i.e.\ removes the function arguments from the stack) \\
    As if with a \hspace{-0.1cm}\St{
        \makebox[0.60cm][l]{mov}
        \makebox[0.60cm][l]{r14}
        \makebox[0.60cm][l]{r13}
        \makebox[0.20cm][l]{0}
    }

    \item Pops a value from the stack to the \St{r13} register
    (that is the value of the stack pointer before the caller's local
    variables) \\
    As if with a \hspace{-0.1cm}\St{
        \makebox[0.60cm][l]{pop}
        \makebox[0.60cm][l]{r13}
        \makebox[0.45cm][l]{0}
    }
\end{itemize}
