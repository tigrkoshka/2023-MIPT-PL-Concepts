\subsection{Notes}

\begin{itemize}
    \item The \St{Karma} processor has a RISC architecture,
    which means that there is no way to operate directly on memory cells,
    all data has to be loaded to the registers before modifications and
    the results have to be explicitly stored back to the memory if necessary
    (see \hyperlink{cmd:data_transfer}{Data transfer commands} for details)

    \item A Von Neumann architecture of the \St{Karma} computer implies that
    both the machine code of the program and the stack are inside the
    global address space.
    The former is placed at its beginning, and the latter starts at its end
    and grows backwards

    \item The stack is \textit{unbounded}, i.e.\ it does not have any size
    limits besides the address space size

    \item Our system allows to write data to any memory cells, including
    the ones occupied by the machine code itself
    Therefore, theoretically, a program might overwrite itself during runtime,
    although such behaviour is considered a bad practice

    \item The same applies to the registers: any register, except for
    \St{flags}, may be passed as an operand to any command
    (that accepts a register operand).
    That includes the \St{r14} and \St{r15} registers, which are meant
    for utility purposes.
    Manually modifying the values of these registers, while allowed,
    is considered a bad practice and can lead to unexpected results
\end{itemize}
