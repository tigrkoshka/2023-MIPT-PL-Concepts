\subsection{Types}

\subsubsection{Basic types}

\vspace{-0.2cm}
There are three architecture-defined basic types in the \St{Karma} assembler:

\begin{itemize}
    \item \St{uint32} is a one-word unsigned integral type
    \item \St{uint64} is a two-word unsigned integral type
    \item \St{double} is a two-word real-valued type
\end{itemize}

\hypertarget{types:two_words_storage}{}
The two-word types (i.e.\ \St{uint64} and \St{double}) are represented by two
consecutive memory cells or registers.
The low 32 bits are placed in the first memory cell/register and the high
32 bits -- in the subsequent one.
When referring to a two-word type value in commands, the specified
memory cell/register is \textit{always} considered to be the first one
(i.e.\ the one containing the low 32 bits of the value).

\newpage

\hypertarget{types:twos_complement}{
    \subsubsection{Source modifier and immediate value: two's complement}
}

The \St{RR} and \St{RI} commands both accept a signed integral operand, which
cannot be straightforwardly represented using the basic types (because there
are no signed integral basic types).

The signed integral values always have the two's complement representation
(see \href{https://en.wikipedia.org/wiki/Two\%27s_complement}{Wiki} for details).
Such a representation is dependant on the bit size of the type that holds the
value.
Therefore, in the binary representation of a command, the same value may
be represented differently if it is a source modifier versus an immediate value.
That occurs iff the value is negative.

Before usage, both the source modifier and the immediate value are transformed
to \St{uint32} using the 32-bit two's complement representation for
\textit{the same value}.
E.g.\ if a source modifier was $-5$ (written in 16-bit two's complement
representation), it will be converted to \textit{the same} $\textit{-5}$
\textit{value}, but in the 32-bit representation.

The two's complement representation of signed values is designed so that it
produces intuitive additive operations results
(see \href{https://en.wikipedia.org/wiki/Two\%27s_complement#Addition}
{explanation}).
Therefore, when specifying the source modifier, one does not need to think about
the signed integral values representation, and may simply assume that
the \textit{signed} source modifier value is added to the \textit{unsigned}
source register value in a purely mathematical sense of things, after which
the resulting value is taken modulo $2^{32}$.

Note, that the addition of the source modifier to the source register value
always happens \textit{before} the main operation (unless, of course,
the command's documentation states that the source modifier is ignored).
Therefore, all \Ss{RI} commands that do not ignore the source modifier
effectively accept two \St{uint32} operands.
