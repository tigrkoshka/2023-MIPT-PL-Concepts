\hypertarget{labels}{
	\subsection{Labels}
}

Before a command or a constant either on the same or on a separate line 
one may place a \textit{label}, which can be used later on in 
the assembler code to indicate the memory address of 
the command or constant it is placed before.

Syntax:

\begin{itemize}
    \item A label must consist only of lowercase latin letters, digits, 
    underscores and dots
    
    \item A label must not start with a digit

    \item A label must be followed by a colon

    \item A label must be the first word in its line
    (it may be the only word of the line)

    \item A label must not conflict with predefined words
    (i.e.\ command names, directives, etc.)

    \item The next word after the label must be a command name\\
    In particular, that means that there cannot be two consecutive
    labels pointing to the same command

    \item The labels must be unique (i.e.\ label redefinition is not allowed)
\end{itemize}

Usage:

\begin{itemize}
    \item A label usage may precede its definition

    \item A label must be defined somewhere in the code to be used,
    there are no predefined labels

    \item A label may only be used as a memory address, i.e.\ only as
    an operand of a command of either \Ss{RM} or \Ss{J} type \\
    Note: this means that, when used, a label is always the last word in
    its line (see \hyperlink{command:formats}{Command formats} section)
\end{itemize}
