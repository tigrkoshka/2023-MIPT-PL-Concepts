\subsection{Directives}
\subsubsection{\St{include} directive}

\vspace{-0.2cm}

An assembler file may have several \St{include} directives at the beginning.
An \St{include} directive must be followed by an unquoted string
representing a relative (to the original file) path to another assembler file.

If such directives are present, the files specified by them are compiled with
the code from the original file as if their contents were copied to
the beginning of the original file in the order of inclusion.
In particular, that means that the code from the original file may use
the labels defined in the included files.
Therefore the labels must be unique throughout all the included files
as well as the original file.

All include statements must appear before the code.
If an \St{include} directive is encountered after the first command
(or label, or constant, whichever occurs first) of the file or if the
relative path specified by the directive is invalid for any reason
(no such file exists, the file cannot be opened, etc.),
a compilation error occurs.

\vspace{-0.35cm}

\hypertarget{directives:end}{
	\subsubsection{\St{end} directive}
}

\vspace{-0.2cm}

An assembler program must have \textit{exactly one} \St{end} directive.

It has one operand which indicates the address of the first instruction
(or a label).
