\subsection{Directives}
\subsubsection{\St{include} directive}

\vspace{-0.2cm}

An assembler file may have several \St{include} directives at the beginning.
An \St{include} directive must be followed by an unquoted string
representing a relative (to the original file) path to another assembler file.
All \St{include} directives must appear before the code.
If an \St{include} directive is encountered after the first command
(or label, or constant, whichever occurs first) of the file or if the
relative path specified by the directive is invalid for any reason
(no such file exists, the file cannot be opened, etc.),
a compilation error occurs.

The directed graph in which the nodes represent the files, and the directed
edges represent the inclusions, drawn from the file in which
the \St{include} directive occurred to the file specified by said directive,
is called the \textit{inclusion tree} of the program.
The program is then said to \textit{consist} of the files represented
by the nodes of its inclusion tree.

A \textit{post-order traversal} of a graph is such a traversal of its nodes
in which the parent node is processed only after all the child nodes have been
processed.
Encountering a node is called \textit{entering} it, if a node has been
encountered, it is said to be \textit{entered}.
After all its children have been processed and the algorithm returned
to its parent, the node is said to be \textit{exited}.

The \St{Karma} compiler uses a \textit{single-entering policy}, which
means that during the traversal each node is only entered once, i.e.\ if
the node was already encountered and at the same time is listed as the child
node of the currently entered, but not exited node, it is simply skipped.
Such a situation can occur either if the \textit{root}
(i.e.\ the node from which the traversal started) is in a cycle or
if a node has several \textit{parents} (i.e.\ nodes from which there is
an edge leading to the current node).
Both these cases are successfully resolved (i.e.\ prevented from causing
an endless loop) by the single-entering policy.

The \textit{order of inclusion} is the order in which the files would be visited
in a post-order traversal of the inclusion tree with the single-entering policy
starting from the root \St{Karma} assembler file
(i.e.\ the one passed to the compiler).

If \St{include} directives are present in a program, the files specified by them
are compiled with the code from the root file as if their contents were
copied to the beginning of the root file in the order of inclusion.
In particular, that means that the code from the root file may use
the labels defined in the included files.
Therefore, the labels must be unique throughout all the included files
as well as the original file.

Note that due to the single-entering policy described above, a valid \St{Karma}
assembler program's inclusion tree may not actually be a tree in the graph
theory sense, i.e.\ it may contain cycles or a file may be included from several
other files.
However, if all includes are \textit{substantial} (i.e.\ if a file is included
only if some labels from it are used in the current file), the cycles are
impossible.

\vspace{-0.35cm}

\hypertarget{directives:end}{
    \subsubsection{\St{end} directive}
}

\vspace{-0.2cm}

An assembler program must have \textit{exactly one} \St{end} directive.

It has one operand which indicates the address of the first instruction
(or a label).
