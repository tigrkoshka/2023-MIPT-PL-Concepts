\subsection{Calculate the factorial of a number using recursion}

\asmsample{

    fact:                 & \# a recursive function calculating the factorial of its argument     \\
    \qquad loadr r0 r14 2 & \# load the first (and only) argument to \St{r0}                      \\
    \qquad cmpi r0 1      & \# compare \St{r0} to \St{1}                                          \\
    \qquad jg skip0       & \# if the argument is greater that \St{1}, recurse                    \\
    \qquad lc r0 1        & \# else store \St{1} (the result for this case, $1! = 1$) to \St{r0}  \\
    \qquad ret 1          & \# return from function and remove the argument from the stack        \\

    &                                                                                             \\

    skip0:                & \# a supplemental function providing recursion                        \\
    \qquad push r0 0      & \# push the current value to the stack ($\star$)                          \\
    \qquad subi r0 1      & \# decrement the current value by \St{1}                              \\
    \qquad push r0 0      & \# push the decremented value to stack as the \St{fact} argument      \\
    \qquad calli fact     & \# \St{r0} contains the result for the decremented value              \\
    \qquad pop r2 0       & \# pop the value stored during the ($\star$) push to \St{r2}              \\
    \qquad mul r0 r2 0    & \# multiply the result for the decremented value by the current value \\
    \qquad ret 1          & \# return from function and remove the argument from the stack        \\

    &                                                                                             \\

    main:                 &                                                                       \\
    \qquad syscall r0 100 & \# read an integer from \St{stdin} to \St{r0}                         \\
    \qquad push r0 0      & \# put \St{r0+0} to the stack as the \St{fact} argument               \\
    \qquad calli fact     & \# call \St{fact}, the function will put the result to \St{r0}        \\
    \qquad syscall r0 102 & \# print \St{r0} to \St{stdout}                                       \\
    \qquad lc r0 10       & \# store the constant \St{10} (`\textbackslash n') to \St{r0}         \\
    \qquad syscall r0 105 & \# print `\textbackslash n' from \St{r0} to \St{stdout}               \\
    \qquad lc r0 0        & \# clear \St{r0}                                                      \\
    \qquad syscall r0 0   & \# exit the program with code \St{0}                                  \\
    end main              & \# start execution from label main                                    \\

}
