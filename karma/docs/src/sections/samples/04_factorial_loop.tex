\subsection{Calculate the factorial of a number using a loop}

\asmsample{

    fact:                    & \# a non-recursive function calculating the factorial of its argument \\
    \qquad loadr r0 r14 2    & \# load the first (and only) argument to \St{r0}                      \\
    \qquad mov r2 r0 0       & \# copy the argument to r2                                            \\
    \qquad lc r0 1           & \# initialise the result with \St{1}                                  \\

    \qquad loop:             & \# a while loop                                                       \\
    \qquad\qquad cmpi r2 1   & \# compare \St{r2} to \St{1}                                          \\
    \qquad\qquad jle out     & \# if the next factor is less or equal to \St{1}, break the cycle     \\
    \qquad\qquad mul r0 r2 0 & \# multiply the current result by the next factor                     \\
    \qquad\qquad subi r2 1   & \# decrement \St{r2} by \St{1}                                        \\
    \qquad\qquad jmp loop    & \# continue the loop                                                  \\

    \qquad out:              & \# out of the while loop                                              \\
    \qquad\qquad ret 1       & \# return from function and remove the argument from the stack        \\

    &                                                                                                \\

    main:                    &                                                                       \\
    \qquad syscall r0 100    & \# read an integer from \St{stdin} to \St{r0}                         \\
    \qquad push r0 0         & \# put \St{r0+0} to the stack as the \St{fact} argument               \\
    \qquad calli fact        & \# call \St{fact}, the function will put the result to \St{r0}        \\
    \qquad syscall r0 102    & \# print \St{r0} to \St{stdout}                                       \\
    \qquad lc r0 10          & \# store the constant \St{10} (`\textbackslash n') to \St{r0}         \\
    \qquad syscall r0 105    & \# print `\textbackslash n' from \St{r0} to \St{stdout}               \\
    \qquad lc r0 0           & \# clear \St{r0}                                                      \\
    \qquad syscall r0 0      & \# exit the program with code \St{0}                                  \\
    end main                 & \# start execution from label main                                    \\

}
