\section{Architecture}

\St{Karma} is a computer with a
\href{https://en.wikipedia.org/wiki/Von_Neumann_architecture}
{Von Neumann architecture} having an address space of $2^{20}$ cells,
one 32-bit \textit{machine word} each.

The processor has 16 one-word (32 bits each) registers \St{r0}-\St{r15},
as well as an additional \St{flags} register (also one-word).
Their usage is described in Table 1.

\hypertarget{registers}{}
{
    \vspace{-0.4cm}
    \renewcommand{\arraystretch}{1.4}
    \begin{table}[h!]
        \centering
        \caption{Usage of \St{Karma} processor registers}
        \vspace{2mm}
        \begin{tabular}{| c | c |}
            \hline
            \St{r0} -- \St{r12} & Free usage                  \\
            	\St{r13}         & Call frame pointer          \\
            \St{r14}         & Stack head pointer          \\
            \St{r15}         & Instruction pointer         \\
            \St{flags}       & Comparison operation result \\
            \hline
        \end{tabular}
    \end{table}
}

\vspace{-0.1cm}

The usage of the \St{r14} register is described in
the \hyperlink{cmd:stack}{Stack-related commands} section, the aim of
the \St{r13} and the \St{r15} registers is discussed in the \hyperlink{cmd:calls}{Function calls}
section, and the usage of the \St{flags} register is detailed in
the \hyperlink{cmd:flags}{Comparisons and jumps} section.
