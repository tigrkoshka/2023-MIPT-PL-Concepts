\section{Code samples}

\subsection{Calculate the square of a number without functions}

{
    \renewcommand{\baselinestretch}{0.5}
    \renewcommand{\arraystretch}{2}

    \selectfont

    \begin{table*}[h!]
        \begin{tabular}{ m{4.5cm}  m{11cm} }

            \St{main:}                  &                                                            \\
            \qquad \St{syscall r0, 100} & ; read an integer from \St{stdin} to \St{r0}               \\
            \qquad \St{mov r2, r0, 0}   & ; copy from \St{r0} to \St{r2}                             \\
            \qquad \St{mul r0, r2, 0}   & ; a pair of registers \St{(r0,r1)} contains the square     \\
            \qquad \St{syscall r0, 102} & ; print from \St{r0} to \St{stdout} (i.e.\ the lower bits) \\
            \qquad \St{lc r0, 10}       & ; store the constant 10 (`\textbackslash n') to \St{r0}    \\
            \qquad \St{syscall r0, 105} & ; print `\textbackslash n' from \St{r0} to \St{stdout}     \\
            \qquad \St{lc r0, 0}        & ; clear \St{r0}                                            \\
            \qquad \St{syscall r0, 0}   & ; exit the program with code \St{0}                        \\
            \qquad \St{end main}        & ; start execution from label main                          \\

        \end{tabular}
    \end{table*}
}

\subsection{Calculate the square of a number with functions}

{
    \renewcommand{\baselinestretch}{0.5}
    \renewcommand{\arraystretch}{2}

    \selectfont

    \begin{table*}[h!]
        \begin{tabular}{ m{4.5cm}  m{11cm} }

            \St{sqr:}                    & ; a function calculating the square with one argument on the stack \\
            \qquad \St{loadr r0, r14, 1} & ; load the first (and only) argument to \St{r0}                    \\
            \qquad \St{mov r2, r0, 0}    & ; copy from \St{r0} to \St{r2}                                     \\
            \qquad \St{mul r0, r2, 0}    & ; a pair of registers \St{(r0,r1)} contains the square             \\
            \qquad \St{ret 1}            & ; return from function and remove the argument from the stack      \\

            &                                                                    \\

            \St{intout:}                 & ; a function printing its argument and `\textbackslash n'          \\
            \qquad \St{load r0, r14, 1}  & ; load the first (and only) argument to \St{r0}                    \\
            \qquad \St{syscall r0, 102}  & ; print \St{r0} to \St{stdout}                                     \\
            \qquad \St{lc r0, 10}        & ; store the constant \St{10} (`\textbackslash n') to \St{r0}       \\
            \qquad \St{syscall r0, 105}  & ; print `\textbackslash n' from \St{r0} to \St{stdout}             \\
            \qquad \St{ret 1}            & ; return from function and remove the argument from the stack      \\

            &                                                                    \\

            \St{main:}                   &                                                                    \\
            \qquad \St{syscall r0, 100}  & ; read an integer from \St{stdin} to \St{r0}                       \\
            \qquad \St{push r0, 0}       & ; put \St{r0+0} to the stack as the \St{sqr} argument              \\
            \qquad \St{calli sqr}        & ; call \St{sqr}, the function will put the result to \St{r0}       \\
            \qquad \St{push r0, 0}       & ; prepare the result of \St{sqr} to be passed to \St{intout}       \\
            \qquad \St{calli intout}     & ; call \St{intout} with the prepared argument                      \\
            \qquad \St{lc r0, 0}         & ; clear \St{r0}                                                    \\
            \qquad \St{syscall r0, 0}    & ; exit the program with code \St{0}                                \\
            \qquad \St{end main}         & ; start execution from label main                                  \\


        \end{tabular}
    \end{table*}
}

\newpage

\subsection{Calculate the factorial of a number using a loop}

{
    \renewcommand{\baselinestretch}{0.5}
    \renewcommand{\arraystretch}{2}

    \selectfont

    \begin{table*}[h!]
        \begin{tabular}{ m{4.5cm}  m{11cm} }

            \St{fact:}                   & ; a non-recursive function calculating the factorial of its argument \\
            \qquad \St{loadr r0, r14, 1} & ; load the first (and only) argument to \St{r0}                      \\
            \qquad \St{mov r2, r0, 0}    & ; copy the argument to r2                                            \\
            \qquad \St{lc r0, 1}         & ; initialise the result with \St{1}                                  \\

            \St{loop:}                   & ; a while loop                                                       \\
            \qquad \St{cmpi r2, 1}       & ; compare \St{r2} to \St{1}                                          \\
            \qquad \St{jle out}          & ; if the next factor is less or equal to \St{1}, break the cycle     \\
            \qquad \St{mul r0, r2, 0}    & ; multiply the current result by the next factor                     \\
            \qquad \St{subi r2, 1}       & ; decrement \St{r2} by \St{1}                                        \\
            \qquad \St{jmp loop}         & ; continue the loop                                                  \\

            \St{out:}                    & ; out of the while loop                                              \\
            \qquad \St{ret 1}            & ; return from function and remove the argument from the stack        \\

            &                                                                      \\

            \St{main:}                   &                                                                      \\
            \qquad \St{syscall r0, 100}  & ; read an integer from \St{stdin} to \St{r0}                         \\
            \qquad \St{push r0, 0}       & ; put \St{r0+0} to the stack as the \St{fact} argument               \\
            \qquad \St{calli fact}       & ; call \St{fact}, the function will put the result to \St{r0}        \\
            \qquad \St{syscall r0, 102}  & ; print \St{r0} to \St{stdout}                                       \\
            \qquad \St{lc r0, 10}        & ; store the constant \St{10} (`\textbackslash n') to \St{r0}         \\
            \qquad \St{syscall r0, 105}  & ; print `\textbackslash n' from \St{r0} to \St{stdout}               \\
            \qquad \St{lc r0, 0}         & ; clear \St{r0}                                                      \\
            \qquad \St{syscall r0, 0}    & ; exit the program with code \St{0}                                  \\
            \qquad \St{end main}         & ; start execution from label main                                    \\


        \end{tabular}
    \end{table*}
}


\subsection{Calculate the factorial of a number using recursion}

{
    \renewcommand{\baselinestretch}{0.5}
    \renewcommand{\arraystretch}{2}

    \selectfont

    \begin{table*}[h!]
        \begin{tabular}{ m{4.5cm}  m{11cm} }

            \St{fact:}                   & ; a recursive function calculating the factorial of its argument     \\
            \qquad \St{loadr r0, r14, 1} & ; load the first (and only) argument to \St{r0}                      \\
            \qquad \St{cmpi r0, 1}       & ; compare \St{r0} to \St{1}                                          \\
            \qquad \St{jg skip0}         & ; if the argument is greater that \St{1}, recurse                    \\
            \qquad \St{lc r0, 1}         & ; else store \St{1} (the result for this case, $1! = 1$) to \St{r0}  \\
            \qquad \St{ret 1}            & ; return from function and remove the argument from the stack        \\

            &                                                                      \\

            \St{skip0:}                  & ; a supplemental function providing recursion                        \\
            \qquad \St{push r0, 0}       & ; push the current value to the stack ($\star$)                      \\
            \qquad \St{subi r0, 1}       & ; decrement the current value by \St{1}                              \\
            \qquad \St{push r0, 0}       & ; push the decremented value to stack as the \St{fact} argument      \\
            \qquad \St{calli fact}       & ; \St{r0} contains the result for the decremented value              \\
            \qquad \St{pop r2, 0}        & ; pop the value stored during the ($\star$) push to \St{r2}          \\
            \qquad \St{mul r0, r2, 0}    & ; multiply the result for the decremented value by the current value \\
            \qquad \St{ret 1}            & ; return from function and remove the argument from the stack        \\

            &                                                                      \\

            \St{main:}                   &                                                                      \\
            \qquad \St{syscall r0, 100}  & ; read an integer from \St{stdin} to \St{r0}                         \\
            \qquad \St{push r0, 0}       & ; put \St{r0+0} to the stack as the \St{fact} argument               \\
            \qquad \St{calli fact}       & ; call \St{fact}, the function will put the result to \St{r0}        \\
            \qquad \St{syscall r0, 102}  & ; print \St{r0} to \St{stdout}                                       \\
            \qquad \St{lc r0, 10}        & ; store the constant \St{10} (`\textbackslash n') to \St{r0}         \\
            \qquad \St{syscall r0, 105}  & ; print `\textbackslash n' from \St{r0} to \St{stdout}               \\
            \qquad \St{lc r0, 0}         & ; clear \St{r0}                                                      \\
            \qquad \St{syscall r0, 0}    & ; exit the program with code \St{0}                                  \\
            \qquad \St{end main}         & ; start execution from label main                                    \\


        \end{tabular}

    \end{table*}
}