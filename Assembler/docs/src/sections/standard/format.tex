\hypertarget{command:formats}{}

\subsection{Command formats}
Each command takes up \textit{exactly one} word, 8 high bits of which specify the operation code and the use of the rest 24 bits is command-specific. With respect to the operation code each command may be of one of the following formats:

    {
    \vspace{-0.4cm}
    \renewcommand{\arraystretch}{1.4}
    \begin{table}[h!]
        \centering
        \caption{\St{Karma} processor command formats}
        \vspace{2mm}
        \centering
        \begin{tabular}{
                >{}                         m{1.2in}
                >{\centering\arraybackslash}m{0.3in}  |
                >{\centering\arraybackslash}m{0.75in} :
                >{\centering\arraybackslash}m{0.75in} :
                >{\centering\arraybackslash}m{0.75in} :
                >{\centering\arraybackslash}m{0.75in} |
                >{}                         m{1.65in}
        }
            \cline{3-6}

            & &

            \textbf{8 bits} & \textbf{4 bits} & \textbf{4 bits} & \textbf{16 bits} & \textbf{Syntax sample} \\

            \cline{3-6}

            Register-memory & \Ss{RM} &
            
            \multirow{4}{*}{
            		\vspace{-0.6cm}\shortstack[c]{command\\ code}
            	} & 
            	register & \multicolumn{2}{c|}{memory address}
            & \St{load r0, 12956} \\

            \cline{4-6}

            Register-register & \Ss{RR} & &
            \vspace{0.2cm}\shortstack[c]{receiver\\ register} & 
            \vspace{0.2cm}\shortstack[c]{source\\ register} & 
            \vspace{0.2cm}\shortstack[c]{source\\ modifier}
            & \St{mov r1, r2, -0xa21} \\

            \cline{4-6}

            Register-immediate & \Ss{RI} & &
            register & \multicolumn{2}{c|}{immediate value} &
            \St{ori r2, 64} \\

            \cline{4-6}

            Jump & \Ss{J} & &
            ignored & \multicolumn{2}{c|}{memory address} &
            \St{calli 01547} \\

            \cline{3-6}
        \end{tabular}
    \end{table}
}

The \textit{memory address} operand for \Ss{RM} and \Ss{J} commands is interpreted as an unsigned integer representing the number of a memory cell (in 0-indexing). The bit size of the operand (20 bits) allows it to represent any memory cell (there are $2^{20}$ of them).

For \Ss{RM} and \Ss{J} commands the \textit{memory address} operand in the assembler code may be represented as:

\begin{itemize}
    \item A decimal number (non-prefixed, not starting with \St{0})
    \item An octal number (with a \St{0} prefix)
    \item A hexadecimal number (with a \St{0x} or a \St{0X} prefix)
\end{itemize}

The same applies to the \textit{source modifier} and \textit{immediate value} operands for \Ss{RR} and \Ss{RI} commands respectively. If the operand is negative, in octal and hexadecimal representations the minus sign is placed before the prefix.

For the sake of not overcomplicating matters all arguments of any command are required.
When using an \Ss{RR} command, if one does not wish to modify the value represented by the source modifier, the \textit{source modifier} should be specified as 0.