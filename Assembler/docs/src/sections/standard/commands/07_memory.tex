\subsubsection{Data transfer commands}

\cmdtable{Data transfer commands} {
%
%            12 & \St{lc} & \Ss{RI} &
%
%            \St{
%                lc
%                r7
%                123
%            } & \St{r7 $=$ 123} \newline
%            Storing immediate operand to the register \newline
%            Supplemental bitwise shift and addition commands are required for storing \newline
%            constants greater than $2^{20} - 1$ \\

%            \hline
%
%            24 & \St{mov} & \Ss{RR} &
%
%            \St{
%                mov
%                r0
%                r3
%                10
%            } &
%
%            Forwarding the value in the source register modified by the immediate \newline
%            operand to the receiver register \newline
%            After the execution of the example above \St{r3+10} is stored in \St{r0} \\

    \hline

    64 & \St{load} & \Ss{RM} & \St{load r0, 12345} & \\

%            Load from memory to register \newline
%            The value stored by the address specified by the immediate operand \newline
%            is copied to the receiver register \newline
%            \St{load r0, 12345} \\

    \hline

    65 & \St{store} & \Ss{RM} & \St{store r0, 12344} & \\

%            Store from register to memory \newline
%            The value stored in the source register is copied to the address \newline
%            specified by the immediate operand \newline
%            \St{store r0, 12344} \\

    \hline

    66 & \St{load2} & \Ss{RM} & \St{load2 r0, 12345} & \\

%            Load two words from memory to registers \newline
%            The value stored by the address specified by the immediate operand and \newline
%            the next memory cell is copied to the receiver register and the next register \newline
%            respectively \newline
%            \St{load2 r0, 12345} \newline
%            In the example above the values from the memory cells \St{12345} and \St{12346} \newline
%            are copied to registers \St{r0} and \St{r1} respectively \\

    \hline

    67 & \St{store2} & \Ss{RM} & \St{store2 r0, 12344} & \\

%            Store two words from registers to memory \newline
%            The value stored in the source register and the next register are copied to \newline
%            the address specified by the immediate operand and the next memory cell \newline
%            respectively \newline
%            \St{store2 r0, 12344} \newline
%            In the example above the values from registers \St{r0} and \St{r1} are copied to \newline
%            the memory cells \St{12344} and \St{12345} respectively \\

    \hline

    68 & \St{loadr} & \Ss{RR} & \St{loadr r0, r1, 15} & \\

%            Load from memory to register \newline
%            The value stored by the address which can be acquired by modifying the \newline
%            source register by the immediate operand is copied to the receiver register \newline
%            \St{loadr r0, r1, 15} \newline
%            In the example above the value from the memory cell \St{r1+15} is copied to \St{r0} \\

    \hline

    69 & \St{storer} & \Ss{RR} & \St{storer r0, r11, 3} & \\

%            Store from register to memory \newline
%            The value stored in the receiver register is copied to the address which can \newline
%            be acquired by modifying the source register by the immediate operand \newline
%            The naming of the argument registers for this command is counter-intuitive: \newline
%            the value is copied \textit{from the receiver register} \newline
%            \St{storer r0, r11, 3} \newline
%            In the example above the value from \St{r0} is copied to the memory cell \St{r11+3} \\

    \hline

    70 & \St{loadr2} & \Ss{RR} & \St{loadr2 r0, r10, 12} & \\

%            Load two words from memory to registers \newline
%            The value stored by the address which can be acquired by modifying \newline
%            the source register by the immediate operand and the next memory cell \newline
%            is copied to the receiver register and the next register respectively \newline
%            \St{loadr2 r0, r10, 12} \newline
%            In the example above the values from the memory cells \St{r10+12} and \St{r10+13} \newline
%            are copied to registers \St{r0} and \St{r1} respectively \\

    \hline

    71 & \St{storer2} & \Ss{RR} & \St{storer2 r0, r3, 10} & \\

%            Store two words from registers to memory \newline
%            The value stored in the receiver register and the next register are copied to \newline
%            the address which can be acquired by modifying the source register by \newline
%            the immediate operand and the next memory cell respectively \newline
%            The naming of the argument registers for this command is counter-intuitive: \newline
%            the value is copied \textit{from the receiver register} \newline
%            \St{storer2 r0, r3, 10} \newline
%            In the example above the values from registers \St{r0} and \St{r1} are copied to \newline
%            the memory cells \St{r3+10} and \St{r3+11} respectively \\
}