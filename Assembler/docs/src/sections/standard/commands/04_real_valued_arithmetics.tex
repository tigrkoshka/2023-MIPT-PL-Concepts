\subsubsection{Real-valued operators}

\cmdtable{Real-valued operators}{
    21 & \St{itod} & \Ss{RR} & \RRcmd{itod}{r2}{r5}{5} &
    \hspace{-0.42cm}
    \St{
        \makebox[1.47cm][l]{(r3,\spp r2)}
        \makebox[0.5cm][c]{$=$}
        \makebox[2cm][l]{double(r5+5)}
    }
    \\

    \hline

    22 & \St{dtoi} & \Ss{RR} & \RRcmd{dtoi}{r2}{r5}{0} &
    \St{r2 = uint32((r6,\spp r5))}
    \\

    \hline


    23 & \St{addd} & \Ss{RR} & \RRcmd{addd}{r2}{r5}{0} &
    \hspace{-0.42cm}
    \St{
        \makebox[1.47cm][l]{(r3,\spp r2)}
        \makebox[0.4cm][c]{$\mathrel{+}=$}
        \makebox[1.47cm][l]{(r6,\spp r5)}
    } \\

    \hline

    24 & \St{subd} & \Ss{RR} & \RRcmd{subd}{r1}{r6}{0} &
    \hspace{-0.42cm}
    \St{
        \makebox[1.47cm][l]{(r2,\spp r1)}
        \makebox[0.4cm][c]{$\mathrel{-}=$}
        \makebox[1.47cm][l]{(r7,\spp r6)}
    } \\

    \hline

    25 & \St{muld} & \Ss{RR} & \RRcmd{muld}{r0}{r2}{0} &
    \hspace{-0.42cm}
    \St{
        \makebox[1.47cm][l]{(r1,\spp r0)}
        \makebox[0.4cm][c]{$\mathrel{\ast}=$}
        \makebox[1.47cm][l]{(r3,\spp r2)}
    } \\

    \hline

    26 & \St{divd} & \Ss{RR} & \RRcmd{divd}{r1}{r3}{0} &
    \hspace{-0.42cm}
    \St{
        \makebox[1.47cm][l]{(r2,\spp r1)}
        \makebox[0.4cm][c]{$\mathrel{/}=$}
        \makebox[1.47cm][l]{(r4,\spp r3)}
    } \\
}

\vspace{0.5cm}

For the \St{double} values storage please refer to \hyperlink{types:two_words_storage}{this section}. For the double values representation please refer to \href{https://en.wikipedia.org/wiki/Floating-point_arithmetic}{Wiki}, specifically to the \href{https://en.wikipedia.org/wiki/Floating-point_arithmetic#Floating-point_numbers}{`Overview'} and \href{https://en.wikipedia.org/wiki/Floating-point_arithmetic#Internal_representation}{`Internal representation'} sections.

\vspace{-0.35cm}

\paragraph{\St{itod}}\

The source modifier is added to the source register in a common manner (see \hyperlink{types:twos_complement}{the respective section} for details), after which the resulting \St{uint32} value is converted to \St{double} and stored in the two registers starting from the specified receiver register.

\vspace{-0.35cm}

\paragraph{\St{dtoi}}\

The source modifier is ignored, but per our agreement must be present for simplicity.

The \St{double} value obtained from the two registers starting from the source register is converted to \St{uint32} and stored in the receiver register. If the value obtained after conversion does not fit into \St{uint32}, an execution error occurs.

\vspace{-0.35cm}

\paragraph{\St{Real-valued arithmetic operators}}\

The source modifier is ignored for all real-valued arithmetic operators, but per our agreement must be present for simplicity.