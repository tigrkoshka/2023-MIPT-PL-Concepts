\section{\St{Karma} assembler standard}

\hypertarget{command:formats}{}

\subsection{Command formats}
Each command takes up \textit{exactly one} word, 8 high bits of which specify the operation code and the use of the rest 24 bits is command-specific. With respect to the operation code each command may be of one of the following formats:

    {
    \vspace{-0.4cm}
    \renewcommand{\arraystretch}{1.4}
    \begin{table}[h!]
        \centering
        \caption{\St{Karma} processor command formats}
        \vspace{2mm}
        \centering
        \begin{tabular}{
                >{}                         m{1.2in}
                >{\centering\arraybackslash}m{0.3in}  |
                >{\centering\arraybackslash}m{0.75in} :
                >{\centering\arraybackslash}m{0.75in} :
                >{\centering\arraybackslash}m{0.75in} :
                >{\centering\arraybackslash}m{0.75in} |
                >{}                         m{1.65in}
        }
            \cline{3-6}

            & &

            \textbf{8 bits} & \textbf{4 bits} & \textbf{4 bits} & \textbf{16 bits} & \textbf{Syntax sample} \\

            \cline{3-6}

            Register-memory & \Ss{RM} &
            
            \multirow{4}{*}{
            		\vspace{-0.6cm}\shortstack[c]{command\\ code}
            	} & 
            	register & \multicolumn{2}{c|}{memory address}
            & \St{load r0, 12956} \\

            \cline{4-6}

            Register-register & \Ss{RR} & &
            \vspace{0.2cm}\shortstack[c]{receiver\\ register} & 
            \vspace{0.2cm}\shortstack[c]{source\\ register} & 
            \vspace{0.2cm}\shortstack[c]{source\\ modifier}
            & \St{mov r1, r2, -0xa21} \\

            \cline{4-6}

            Register-immediate & \Ss{RI} & &
            register & \multicolumn{2}{c|}{immediate value} &
            \St{ori r2, 64} \\

            \cline{4-6}

            Jump & \Ss{J} & &
            ignored & \multicolumn{2}{c|}{memory address} &
            \St{calli 01547} \\

            \cline{3-6}
        \end{tabular}
    \end{table}
}

The \textit{memory address} operand for \Ss{RM} and \Ss{J} commands is interpreted as an unsigned integer representing the number of a memory cell (in 0-indexing). The bit size of the operand (20 bits) allows it to represent any memory cell (there are $2^{20}$ of them).

For \Ss{RM} and \Ss{J} commands the \textit{memory address} operand in the assembler code may be represented as:

\begin{itemize}
    \item A decimal number (non-prefixed, not starting with \St{0})
    \item An octal number (with a \St{0} prefix)
    \item A hexadecimal number (with a \St{0x} or a \St{0X} prefix)
\end{itemize}

The same applies to the \textit{source modifier} and \textit{immediate value} operands for \Ss{RR} and \Ss{RI} commands respectively. If the operand is negative, in octal and hexadecimal representations the minus sign is placed before the prefix.

For the sake of not overcomplicating matters all arguments of any command are required.
When using an \Ss{RR} command, if one does not wish to modify the value represented by the source modifier, the \textit{source modifier} should be specified as 0.
\newpage

\subsection{Command set}

\newcommand{\cmdtable}[2]{
        {
        \vspace{-0.7cm}
        \renewcommand{\arraystretch}{1.4}
        \begin{table}[h!]
            \centering
            \caption{#1}
            \vspace{2mm}
            \centering
            \begin{tabular}{|
                    >{\centering\arraybackslash} m{1.0cm} |
                    >{\centering\arraybackslash} m{1.8cm} |
                    >{\centering\arraybackslash} m{1.3cm} |
                    >{}                          m{3.8cm} |
                    >{}                          m{4.8cm} |
            }

                \hline

                Code & Name & Format & Syntax sample & Description \\

                \hline

                #2

                \hline

            \end{tabular}
        \end{table}
    }
    \vspace{-0.5cm}
}

\newcommand{\RMcmd}[2]{
    \St{
        \makebox[0.85cm][l]{#1}
        \makebox[0.45cm][l]{#2}
    }
}

\newcommand{\RRcmd}[4]{
    \St{
        \makebox[0.85cm][l]{#1}
        \makebox[0.45cm][l]{#2}
        \makebox[0.45cm][l]{#3}
        \makebox[0.20cm][l]{#4}
    }
}

\newcommand{\RIcmd}[3]{
    \St{
        \makebox[0.85cm][l]{#1}
        \makebox[0.45cm][l]{#2}
        \makebox[0.45cm][l]{#3}
    }
}

\newcommand{\Jcmd}[2]{
    \St{
        \makebox[0.85cm][l]{#1}
        \makebox[0.45cm][l]{#2}
    }
}

\subsubsection{System commands}

\cmdtable{System commands}{
    0 & \St{halt} & \Ss{RI} &

    \St{halt r1 0} & Stop processor \\

    \hline

    1 & \St{syscall} & \Ss{RI} &

    \St{syscall r0 100} & System call \\
}

\paragraph{\St{halt}}\

The \St{halt} command sends an interruption signal to the CPU, which halts it
until the next external signal is received, after which the execution is continued.
The immediate value operand is ignored.

\vspace{-0.35cm}

\paragraph{\St{syscall}}\

The \St{syscall} command's immediate operand specifies the call code.
The semantics of those codes is described in Table 4:

    {
    \vspace{-0.4cm}
    \renewcommand{\arraystretch}{1.4}
    \begin{table}[h!]
        \centering
        \caption{System call codes}
        \vspace{2mm}
        \begin{tabular}{|
                >{\centering\arraybackslash} m{1.0cm} |
                >{\centering\arraybackslash} m{2.3cm} |
                >{}                          m{8cm}   |
                >{}                          m{3cm}   |
        }
            \hline
            Code & Name             & Description                                           & Register operand  \\
            \hline
            0    & \St{EXIT}        & Finish execution without error                        & ---               \\
            100  & \St{SCANINT}     & Get an integer value from \St{stdin}                  & Receiver          \\
            101  & \St{SCANDOUBLE}  & Get a floating-point value from \St{stdin}            & Low-bits receiver \\
            102  & \St{PRINTINT}    & Output an integer value to \St{stdout}                & Source            \\
            103  & \St{PRINTDOUBLE} & Output a floating-point value to \St{stdout}          & Low-bits source   \\
            104  & \St{GETCHAR}     & Get a single \St{ASCII} character from \St{stdin}     & Receiver          \\
            105  & \St{PUTCHAR}     & Output a single \St{ASCII} character from \St{stdout} & Source            \\
            \hline
        \end{tabular}
    \end{table}
}

Specifying a value for the \St{syscall} immediate operand not listed in
the table above leads to an error during compilation or execution
(depending on whether such a value was originally specified in an assembler
code or in a handcrafted executable file).

If the source register for the \St{PUTCHAR} system call contains a value greater
than 255 (that is, not representing an \St{ASCII} character), an execution error
will occur.


\newpage

\subsubsection{Integer arithmetic operators}

\cmdtable{Integer arithmetic operators}{
    2 & \St{add} & \Ss{RR} & \RRcmd{add}{r1}{r2}{-3} &
    \St{r1\spp $\mathrel{+}=$\spp r2\spp -\spp 3} \\

    \hline

    3 & \St{addi} & \Ss{RI} & \RIcmd{addi}{r4}{10} &
    \St{r4\spp $\mathrel{+}=$\spp 10} \\

    \hline

    4 & \St{sub} & \Ss{RR} & \RRcmd{sub}{r3}{r5}{5} &
    \St{r3\spp $\mathrel{-}=$\spp r5\spp +\spp 5} \\

    \hline

    5 & \St{subi} & \Ss{RI} & \RIcmd{subi}{r4}{1} &
    \St{r4\spp $\mathrel{-}=$\spp 1} \\

    \hline

    6 & \St{mul} & \Ss{RR} & \RRcmd{mul}{r3}{r7}{-2} &
    \hspace{-0.20cm}
    \St{(r4,\spp r3)\spp $=$\spp r3\spp $\ast$\spp (r7\spp -\spp 2)} \\

    \hline

    7 & \St{muli} & \Ss{RI} & \RIcmd{muli}{r5}{100} &
    \hspace{-0.20cm}
    \St{(r6,\spp r5)\spp $=$\spp r5\spp $\ast$\spp 100} \\

    \hline

    8 & \St{div} & \Ss{RR} & \RRcmd{div}{r3}{r8}{5} &
    \vspace{0.18cm}
    \hspace{-0.34cm}
    \St{
        \shortstack[l]{
            tmp\spp $=$\spp (r4,\spp r3) \\
            r3\spp $=$\spp tmp\spp /\spp (r8\spp +\spp 5) \\
            r4\spp $=$\spp tmp\spp \%\spp (r8\spp +\spp 5)
        }
    }\vspace{0.03cm}
    \\

    \hline

    9 & \St{divi} & \Ss{RI} & \RIcmd{divi}{r6}{7} &
    \vspace{0.18cm}
    \St{
        \shortstack[l]{
            tmp\spp $=$\spp (r7,\spp r6) \\
            r6\spp $=$\spp tmp\spp /\spp 7 \\
            r7\spp $=$\spp tmp\spp \%\spp 7
        }
    }
    \vspace{0.03cm}
    \\
}

\paragraph{\St{add}, \St{addi}, \St{sub}, \St{subi}}\

These commands take two \St{uint32} values obtained from the specified operands (see \hyperlink{types:twos_complement}{the respective section} for details) and produce a \St{uint32} value as a result.

In the mathematical sense of things, for the unsigned (as-is) interpretation of the operands the arithmetic is simply performed modulo $2^{32}$. E.g.\ if \St{r0} contained 0, after \St{subi\sppp r0\sppp 1}, the value in \St{r0} will be $2^{32}-1$.

At the same time, if both terms of each operation were to be interpreted as \textit{signed} integral values in the 32-bit two's complement representation, the result, if interpreted as a \textit{signed} integral value in the same representation, would be mathematically correct.
That is provided by the two's complement representation of the signed operands (see explanations for \href{https://en.wikipedia.org/wiki/Two\%27s_complement#Addition}{addition} and \href{https://en.wikipedia.org/wiki/Two\%27s_complement#Subtraction}{subtraction}).

\vspace{-0.35cm}

\paragraph{\St{mul}, \St{muli}}\

These commands multiply two \St{uint32} values obtained from the specified operands (see \hyperlink{types:twos_complement}{the respective section} for details), produce a \St{uint64} value and store it in the two registers starting from the specified receiver register (see \hyperlink{types:two_words_storage}{here} for details).

Unlike additive operations, these operations would not produce the correct results if the factors were to interpreted as \textit{signed} integral values in the 32-bit two's complement representation. That is because in order to produce valid results in such a case, the factors must first be \textit{extended} to the size of the destination type, i.e.\ rewritten in the 64-bit two's complement representation (see \href{https://en.wikipedia.org/wiki/Two\%27s_complement#Multiplication}{explanation}). However, if we were to perform such an extension, we would block the opportunity to multiply two big \textit{unsigned} values (because we would have interpreted them as \textit{signed} values and added non-zero bits to the right of the original representation when extending, thus producing an incorrect result for this case).

Overall, one must be cautious when specifying the source modifier and the immediate value in these operations, and bear in mind that if those produce negative factors, they will firstly be converted to \St{uint32} by being taken modulo $2^{32}$.

\vspace{-0.35cm}

\paragraph{\St{div}, \St{divi}}\

These commands accept a \St{uint64} dividend from the two registers starting from the specified receiver register (see \hyperlink{types:two_words_storage}{here} for storage details) and a \St{uint32} divisor and perform an integral division. The quotient is then placed in the receiver register and the remainder -- in the next register.

If the quotient does not fit into a register, a `quotient overflow' execution error occurs.

Like the \St{mul} and \St{muli} commands, these command only work with \textit{unsigned} operands and would not produce valid results if the operands were to be interpreted as \textit{signed} integral values.

\newpage

\subsubsection{Bitwise operators}

\cmdtable{Bitwise operators}{
    10 & \St{not} & \Ss{RI} & \RIcmd{not}{r1}{0} &
    \St{$\text{r1} = \text{\textasciitilde r1}$} \\

    \hline

    11 & \St{shl} & \Ss{RR} & \RRcmd{shl}{r1}{r2}{1} &
    \St{$\text{r1} \mathrel{\ll}= \text{r2} + \text{1}$} \\

    \hline

    12 & \St{shli} & \Ss{RI} & \RIcmd{shli}{r1}{2} &
    \St{$\text{r1} \mathrel{\ll}= \text{2}$} \\

    \hline

    13 & \St{shr} & \Ss{RR} & \RRcmd{shr}{r1}{r2}{-4} &
    \St{$\text{r1} \mathrel{\gg}= \text{r2} - \text{4}$} \\

    \hline

    14 & \St{shri} & \Ss{RI} & \RIcmd{shri}{r1}{2} &
    \St{$\text{r1} \mathrel{\gg}= \text{2}$} \\

    \hline

    15 & \St{and} & \Ss{RR} & \RRcmd{and}{r4}{r6}{3} &
    \St{$\text{r4} \mathrel{\&}= \text{r6} + \text{3}$} \\

    \hline

    16 & \St{andi} & \Ss{RI} & \RIcmd{andi}{r5}{2} &
    \St{$\text{r5} \mathrel{\&}= \text{2}$} \\

    \hline

    17 & \St{or} & \Ss{RR} & \RRcmd{or}{r3}{r2}{-2} &
    \St{$\text{r3} \mathrel{|}= \text{r2} - \text{2}$} \\

    \hline

    18 & \St{ori} & \Ss{RI} & \RIcmd{ori}{r6}{100} &
    \St{$\text{r6} \mathrel{|}= \text{100}$} \\

    \hline

    19 & \St{xor} & \Ss{RR} & \RRcmd{xor}{r1}{r5}{0} &
    \St{$\text{r1} \mathrel{^\wedge}= \text{r5}$} \\

    \hline

    20 & \St{xori} & \Ss{RI} & \RIcmd{xori}{r1}{127} &
    \St{$\text{r1} \mathrel{^\wedge}= \text{127}$} \\
}
\paragraph{\St{not}}\

The immediate value is ignored, but per our agreement must be present for simplicity.

\vspace{-0.35cm}
\paragraph{Other bitwise operators}\

All other bitwise operators take two \St{uint32} values obtained from
the specified operands in the \hyperlink{types:twos_complement}{common manner}
and produce a \St{uint32} value as a result.
The right hand side operand of the bitwise shift operators must be less than
the size of a machine word (i.e.\ no more than 31), otherwise an execution
error occurs.


\subsubsection{Real-valued operators}

\cmdtable{Real-valued operators}{
    21 & \St{itod} & \Ss{RR} & \RRcmd{itod}{r2}{r5}{5} &
    \St{$(\text{r3}, \text{r2}) = \text{double}(\text{r5} + \text{5})$}
    \\

    \hline

    22 & \St{dtoi} & \Ss{RR} & \RRcmd{dtoi}{r2}{r5}{0} &
    \St{$\text{r2} = \text{uint32}((\text{r6}, \text{r5}))$}\\

    \hline


    23 & \St{addd} & \Ss{RR} & \RRcmd{addd}{r2}{r5}{0} &
    \St{$(\text{r3}, \text{r2}) \mathrel{+}= (\text{r6}, \text{r5})$} \\

    \hline

    24 & \St{subd} & \Ss{RR} & \RRcmd{subd}{r1}{r6}{0} &
    \St{$(\text{r2}, \text{r1}) \mathrel{-}= (\text{r7}, \text{r6})$} \\

    \hline

    25 & \St{muld} & \Ss{RR} & \RRcmd{muld}{r0}{r2}{0} &
    \St{$(\text{r1}, \text{r0}) \mathrel{\ast}= (\text{r3}, \text{r2})$} \\

    \hline

    26 & \St{divd} & \Ss{RR} & \RRcmd{divd}{r1}{r3}{0} &
    \St{$(\text{r2}, \text{r1}) \mathrel{/}= (\text{r4}, \text{r3})$} \\
}

\vspace{0.5cm}

For the \St{double} values storage please refer to
\hyperlink{types:two_words_storage}{this section}.
For the double values memory representation please refer to
\href{https://en.wikipedia.org/wiki/Floating-point_arithmetic}{Wiki},
specifically to the
\href{https://en.wikipedia.org/wiki/Floating-point_arithmetic#Floating-point_numbers}
{Overview} and
\href{https://en.wikipedia.org/wiki/Floating-point_arithmetic#Internal_representation}
{Internal representation} sections.

\vspace{-0.35cm}
\paragraph{\St{itod}}\

The source modifier is added to the source register in
the \hyperlink{types:twos_complement}{common manner}, after which the resulting
\St{uint32} value is converted to \St{double} and stored into the two registers
starting from the specified receiver register.

\vspace{-0.35cm}
\paragraph{\St{dtoi}}\

The source modifier is ignored, but per our agreement must be present for
simplicity.

The \St{double} value obtained from the two registers starting from the source
register is converted to \St{uint32} by discarding the fractional part
and stored into the receiver register.
If the resulting value does not fit into a register, an execution error occurs.

\vspace{-0.35cm}
\paragraph{\St{Real-valued arithmetic operators}}\

The source modifier is ignored for all real-valued arithmetic operators,
but per our agreement must be present for simplicity.


\newpage

\subsubsection{Stack operators and function calls}

\cmdtable{Stack operators and function calls}{
    38 & \St{push} & \Ss{RI} & \RIcmd{push}{r0}{255} & fff \\

%    Push the value from the source register modified by the immediate operand \newline
%    to the stack (and then move the stack pointer) \newline
%    \St{push r0, 255} \newline
%    In the example above the value \St{r0+255} is stored to the address from \St{r14}, \newline
%    after which the stack pointer (\St{r14}) is decremented by 1 

    \hline

    39 & \St{pop} & \Ss{RI} & \RIcmd{pop}{r3}{3} & fff \\

%    Pop the value from the stack and store it in the receiver register after \newline
%    modifying by the immediate operand (after moving the stack pointer) \newline
%    \St{pop r3, 3} \newline
%    In the example above the stack pointer (\St{r14}) is incremented by 1, \newline
%    after which the value stored by the address from \St{r14} is incremented by \St{3} \newline
%    and stored in \St{r3} & \\

    \hline

    40 & \St{call} & \Ss{RR} & \RRcmd{call}{r0}{r5}{2} & fff \\

%    Call the function, the address of which can be acquired by modifying \newline
%    the source register by the immediate operand \newline
%    The address of the command following the current one is stored \newline
%    in the receiver register \newline
%    \St{call r0, r5, 2} \newline
%    In the example above the function stored by the address \St{r5+2} is called and \newline
%    the address of the command following the current one is both pushed to the \newline
%    stack (i.e.\ stored by the address from \St{r14} with a consequent decrement of \newline
%    \St{r14}) and stored in \St{r0} & \\

    \hline

    41 & \St{calli} & \Ss{J} & \St{calli\kern 0.5em 21913} & fff \\

%    Call the function, the address of which is specified by the immediate operand \newline
%    \St{calli 13323} \newline
%    In the example above the function stored by the address \St{13323} is called \newline
%    and the address of the command following the current one is pushed to \newline
%    the stack (i.e.\ stored by the address from \St{r14} with a consequent \newline
%    decrement of \St{r14}) & \\

    \hline

    42 & \St{ret} & \Ss{J} & \St{ret\sppp 3} & fff \\

%    Return from function to caller \newline
%    The address of the next executed instruction is popped from the stack \newline
%    The immediate operand specifies the number of additional words that should \newline
%    be ejected from the stack (by simply incrementing the \St{r14} pointer) before \newline
%    popping the next executed instruction address (it must equal the number \newline
%    of the function arguments) \newline
%    \St{ret 3} \newline
%    In the example above the pointer from \St{r14} is incremented by $3 + 1 = 4$, \newline
%    after which the next executed instruction address is acquired by the address \newline
%    from \St{r14} & \\
}

\newpage

\subsubsection{Comparisons and jumps}

\cmdtable{Comparisons and jumps} {
    43 & \St{cmp} & \Ss{RR} & \RRcmd{cmp}{r0}{r1}{2} & \St{r0\spp $<=>$\spp r1\spp +\spp 2} \\

    \hline

    44 & \St{cmpi} & \Ss{RI} & \RIcmd{cmpi}{r0}{0} & \St{r0\spp $<=>$\spp 0}  \\

    \hline

    45 & \St{cmpd} & \Ss{RR} & \RRcmd{cmpd}{r1}{r4}{0} &
    \hspace{-0.42cm}
    \St{
        \makebox[1.47cm][l]{(r2,\spp r1)}
        \makebox[0.6cm][c]{$<=>$}
        \hspace{-0.3cm}
        \makebox[1.47cm][l]{(r6,\spp r5)}
    } \\

    \hline

    46 & \St{jmp} & \Ss{J} & \Jcmd{jmp}{0xffa} & Unconditional jump \\

    \hline

    47 & \St{jne} & \Ss{J} & \Jcmd{jne}{0xd471} & Jump if not equal \\

    \hline

    48 & \St{jeq} & \Ss{J} & \Jcmd{jeq}{05637} & Jump if equal \\

    \hline

    49 & \St{jle} & \Ss{J} & \Jcmd{jle}{1234} & Jump if less or equal \\

    \hline

    50 & \St{jl} & \Ss{J} & \Jcmd{jl}{0X1e4b} & Jump if less \\

    \hline

    51 & \St{jge} & \Ss{J} & \Jcmd{jge}{29834} & Jump if greater or equal \\

    \hline

    52 & \St{jg} & \Ss{J} & \Jcmd{jg}{03457} & Jump if greater \\
}

\paragraph{Flags register}\

To allow for basic execution branching, an additional \St{flags} register is supported.
It holds the result of the latest comparison.
Only the lowest 6 bits of this register are used.
The semantics of those bits is described in \hyperlink{flags:bits}{Table 11}.

\hypertarget{flags:bits}{}
{
    \vspace{-0.4cm}
    \renewcommand{\arraystretch}{1.4}
    \begin{table}[h!]
        \centering
        \caption{\St{flags} register bits semantics (low-to-high, 0-indexed)}
        \vspace{2mm}
        \begin{tabular}{| c | c |}
            \hline
            0 & Equal            \\
            1 & Not equal        \\
            2 & Greater          \\
            3 & Less             \\
            4 & Greater or equal \\
            5 & Less or equal    \\
            \hline
        \end{tabular}
    \end{table}
}

Several bits may be simultaneously filled.
For example, if the latest comparison resulted in equality, the value of the \St{flags} register will be $110001_2$, because equality causes the `less/greater or equal' conditions to also be true.

\newpage

\subsubsection{Data transfer commands}

\cmdtable{Data transfer commands} {
%
%            12 & \St{lc} & \Ss{RI} &
%
%            \St{
%                lc
%                r7
%                123
%            } & \St{r7 $=$ 123} \newline
%            Storing immediate operand to the register \newline
%            Supplemental bitwise shift and addition commands are required for storing \newline
%            constants greater than $2^{20} - 1$ \\

%            \hline
%
%            24 & \St{mov} & \Ss{RR} &
%
%            \St{
%                mov
%                r0
%                r3
%                10
%            } &
%
%            Forwarding the value in the source register modified by the immediate \newline
%            operand to the receiver register \newline
%            After the execution of the example above \St{r3+10} is stored in \St{r0} \\

    \hline

    64 & \St{load} & \Ss{RM} & \St{load r0, 12345} & \\

%            Load from memory to register \newline
%            The value stored by the address specified by the immediate operand \newline
%            is copied to the receiver register \newline
%            \St{load r0, 12345} \\

    \hline

    65 & \St{store} & \Ss{RM} & \St{store r0, 12344} & \\

%            Store from register to memory \newline
%            The value stored in the source register is copied to the address \newline
%            specified by the immediate operand \newline
%            \St{store r0, 12344} \\

    \hline

    66 & \St{load2} & \Ss{RM} & \St{load2 r0, 12345} & \\

%            Load two words from memory to registers \newline
%            The value stored by the address specified by the immediate operand and \newline
%            the next memory cell is copied to the receiver register and the next register \newline
%            respectively \newline
%            \St{load2 r0, 12345} \newline
%            In the example above the values from the memory cells \St{12345} and \St{12346} \newline
%            are copied to registers \St{r0} and \St{r1} respectively \\

    \hline

    67 & \St{store2} & \Ss{RM} & \St{store2 r0, 12344} & \\

%            Store two words from registers to memory \newline
%            The value stored in the source register and the next register are copied to \newline
%            the address specified by the immediate operand and the next memory cell \newline
%            respectively \newline
%            \St{store2 r0, 12344} \newline
%            In the example above the values from registers \St{r0} and \St{r1} are copied to \newline
%            the memory cells \St{12344} and \St{12345} respectively \\

    \hline

    68 & \St{loadr} & \Ss{RR} & \St{loadr r0, r1, 15} & \\

%            Load from memory to register \newline
%            The value stored by the address which can be acquired by modifying the \newline
%            source register by the immediate operand is copied to the receiver register \newline
%            \St{loadr r0, r1, 15} \newline
%            In the example above the value from the memory cell \St{r1+15} is copied to \St{r0} \\

    \hline

    69 & \St{storer} & \Ss{RR} & \St{storer r0, r11, 3} & \\

%            Store from register to memory \newline
%            The value stored in the receiver register is copied to the address which can \newline
%            be acquired by modifying the source register by the immediate operand \newline
%            The naming of the argument registers for this command is counter-intuitive: \newline
%            the value is copied \textit{from the receiver register} \newline
%            \St{storer r0, r11, 3} \newline
%            In the example above the value from \St{r0} is copied to the memory cell \St{r11+3} \\

    \hline

    70 & \St{loadr2} & \Ss{RR} & \St{loadr2 r0, r10, 12} & \\

%            Load two words from memory to registers \newline
%            The value stored by the address which can be acquired by modifying \newline
%            the source register by the immediate operand and the next memory cell \newline
%            is copied to the receiver register and the next register respectively \newline
%            \St{loadr2 r0, r10, 12} \newline
%            In the example above the values from the memory cells \St{r10+12} and \St{r10+13} \newline
%            are copied to registers \St{r0} and \St{r1} respectively \\

    \hline

    71 & \St{storer2} & \Ss{RR} & \St{storer2 r0, r3, 10} & \\

%            Store two words from registers to memory \newline
%            The value stored in the receiver register and the next register are copied to \newline
%            the address which can be acquired by modifying the source register by \newline
%            the immediate operand and the next memory cell respectively \newline
%            The naming of the argument registers for this command is counter-intuitive: \newline
%            the value is copied \textit{from the receiver register} \newline
%            \St{storer2 r0, r3, 10} \newline
%            In the example above the values from registers \St{r0} and \St{r1} are copied to \newline
%            the memory cells \St{r3+10} and \St{r3+11} respectively \\
}

\newpage

\subsection{Further specifications}

\hypertarget{float:storage}{}

\subsubsection{Floating-point values}

Floating-point values are represented in base-2 scientific notation, i.e.\ in the form $m\cdot 2^n$, \newline
where $m\in [1, 2)$ is called a \textit{mantissa} and $n\in \mathbb{Z}$ -- an \textit{exponent}.

In memory they have a 64-bit representation.
The meaning of those bits (high to low) is as follows:

\begin{itemize}
    \item 1 bit -- sign (0 means $+$, 1 means $-$)
    \item 11 bits -- the exponent incremented by 1023
    \item 52 bits -- the fractional part of the mantissa
\end{itemize}

\hypertarget{syscall:details}{}

\subsubsection{System calls}

The \hyperlink{syscall}{\St{syscall}} operation has an immediate operand which specifies the call code.
The semantics of those codes is described in \hyperlink{syscall:codes}{Table 3}.

\hypertarget{syscall:codes}{}
{
    \renewcommand{\arraystretch}{1.4}
    \begin{table}[h!]
        \centering
        \caption{System call codes}
        \vspace{2mm}
        \begin{tabular}{| c | c | c | c |}
            \hline
            Code & Name             & Description                                           & Register operand  \\
            \hline
            0    & \St{EXIT}        & Finish execution without error                        & --                \\
            100  & \St{SCANINT}     & Get an integer value from \St{stdin}                  & Receiver          \\
            101  & \St{SCANDOUBLE}  & Get a floating-point value from \St{stdin}            & Low-bits receiver \\
            102  & \St{PRINTINT}    & Output an integer value to \St{stdout}                & Source            \\
            103  & \St{PRINTDOUBLE} & Output a floating-point value to \St{stdout}          & Low-bits source   \\
            104  & \St{GETCHAR}     & Get a single \St{ASCII} character from \St{stdin}     & Receiver          \\
            105  & \St{PUTCHAR}     & Output a single \St{ASCII} character from \St{stdout} & Source            \\
            \hline
        \end{tabular}
    \end{table}
}

Notes:

\begin{itemize}
    \item The floating-point value storage convention is the same as for the \hyperlink{addd}{\St{addd}} command, i.e.\ the specified register holds the lower bits of the value, and the next register holds the higher bits.
    \item If the register provided for the \St{PUTCHAR} system call holds a value greater than 255, an error occurs
    \item If the \St{syscall} command receives an unknown code, an error occurs
\end{itemize}

\hypertarget{flags:details}{}

\subsubsection{Flags}

To allow for basic execution branching, an additional \St{flags} register is supported.
It holds the result of the latest comparison.
Only the lowest 6 bits of this register are used.
The semantics of those bits is described in \hyperlink{flags:bits}{Table 4}.

\hypertarget{flags:bits}{}
{
    \renewcommand{\arraystretch}{1.4}
    \begin{table}[h!]
        \centering
        \caption{\St{flags} register bits semantics (counting from the lowest, 0-indexed)}
        \vspace{2mm}
        \begin{tabular}{| c | c |}
            \hline
            0 & Equal            \\
            1 & Not equal        \\
            2 & Greater          \\
            3 & Less             \\
            4 & Greater or equal \\
            5 & Less or equal    \\
            \hline
        \end{tabular}
    \end{table}
}

Several bits may be simultaneously filled.
For example, if the latest comparison resulted in equality, the value of the \St{flags} register will be $110001_2$, because equality causes the `less/greater or equal' conditions to also be true.

\subsubsection{Labels}

Either before a command or on a separate line one may place a \textit{label}, which can be used later on in the assembler code to indicate the address of the command it is placed before.

Syntax:

\begin{itemize}
	\item A label must consist only of lowercase latin letters and/or digits and not start with a digit
	\item A label must be followed by a colon
    \item A label must be the first word in its line (it may be the only word of the line)
    \item A label must not conflict with predefined words (i.e.\ command names, directives, etc.)
	\item The labels must be unique (i.e. label redefinition is not allowed)
\end{itemize}

Usage:

\begin{itemize}
	\item A label usage may precede its definition
    \item A label must be defined somewhere in the code to be used, there are no predefined labels
    \item A label may only be used as a memory address, i.e.\ only in command of either \Ss{RM} or \Ss{J} type\\
    Note: this means that, when used, a label is always the last word in its line (see \hyperlink{command:formats}{command formats})
\end{itemize}

\subsubsection{\St{End} directive}

An assembler program must have \textit{exactly one} \St{end} directive, which must be in \textit{the last} line of the program.
It has one operand which indicates the address of the first instruction (or a label).

\subsubsection{Comments}

Each line may contain a semicolon.
If it does, everything after the semicolon is considered a comment and is not compiled into the executable file.
Multiline comments are not allowed.

\vspace{.4in}

\subsection{Notes}

\begin{itemize}
    \item The \St{Karma} processor has a RISC architecture, which means that there is no way to operate directly on memory cells, all data has to be loaded to the registers before modifications and the results have to be explicitly stored back to the memory if necessary
    \item A function call does not include the function arguments.
          They can be passed either via the stack or via the registers (by a programmer-defined convention).
          However, if a function is directly or indirectly recursive, the best practice is to pass the arguments via the stack
    \item A Von Neumann architecture of the \St{Karma} computer implies that both the machine code of the program and the stack are inside the global address space.
          The machine code is placed at its beginning, while the stack starts at its end and grows `backwards'
    \item The stack does not have any size limits besides the address space size
    \item Our system allows to write data to any memory cells, including the ones occupied by the machine code itself.
          Therefore, theoretically, a program might overwrite itself during runtime, although such behaviour is not considered a good practice
\end{itemize}
