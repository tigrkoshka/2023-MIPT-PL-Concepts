\hypertarget{command:formats}{}

\subsection{Command formats}
With respect to the operation code each command may be of one of the following formats:

    {
    \renewcommand{\arraystretch}{1.4}
    \begin{table}[h!]
        \centering
        \caption{\St{Karma} processor command formats}
        \vspace{2mm}
        \centering
        \begin{tabular}{
                >{}                         m{1.2in}
                >{}                         m{0.3in}  |
                >{\centering\arraybackslash}m{0.75in} :
                >{\centering\arraybackslash}m{0.75in} :
                >{\centering\arraybackslash}m{0.75in} :
                >{\centering\arraybackslash}m{0.75in} |
                >{}                         m{1.5in}
        }
            \cline{3-6}

            & &

            8 bits & 4 bits & 4 bits & 16 bits & Example: \\

            \cline{3-6}

            Register-memory & \Ss{RM} &
            command & register & \multicolumn{2}{c|}{memory address}
            & \St{load r0, 12956} \\

            \cline{3-6}

            Register-register & \Ss{RR} &
            command & \ receiver\newline register & \ source\newline register & \ \ source\newline modifier
            & \St{mov r1, r2, -0xa21} \\

            \cline{3-6}

            Register-immediate & \Ss{RI} &
            command & register & \multicolumn{2}{c|}{immediate operand}
            & \St{ori r2, 64} \\

            \cline{3-6}

            Jump & \Ss{J} &
            command & ignored & \multicolumn{2}{c|}{memory address}
            & \St{calli 01547} \\

            \cline{3-6}
        \end{tabular}
    \end{table}
}

For \Ss{RM} and \Ss{J} format commands the address operand in the code may be:

\begin{itemize}
    \item A decimal number (non-prefixed, not starting with 0)
    \item An octal number (with a \St{0} prefix)
    \item A hexadecimal number (with a \St{0x} or a \St{0X} prefix)
\end{itemize}

The same applies to the immediate operand for \Ss{RR} and \Ss{RI} format commands.

For the sake of not overcomplicating matters all arguments of any command are required.
If one does not need the source modifier in a command of the \Ss{RR} format, the immediate operand should be specified as 0.