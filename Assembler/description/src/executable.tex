\section{\St{Karma} executable file}

To run a program on a \St{Karma} computer one needs to generate an executable file which contains meta-information about the machine code and the code itself.
The executable file is stored in a remote storage (e.g.\ a hard drive or an SSD) as a byte sequence.
The header of the executable file takes up exactly 512 bytes.
The format of the executable file is described in \hyperlink{flags:bits}{Table 5}.

\hypertarget{executable:format}{}
{
    \renewcommand{\arraystretch}{1.4}
    \begin{table}[h!]
        \centering
        \caption{\St{Karma} executable file format}
        \vspace{2mm}
        \begin{tabular}{| c | c |}
            \hline
            Bytes  & Contents                            \\
            \hline
            0..15  & \St{ASCII} string ``ThisIsKarmaExec'' \\
            16..19 & Program code size                     \\
            20..23 & Program constants size                \\
            24..27 & Program data size                     \\
            28..31 & Address of the first instruction      \\
            32..35 & Initial stack pointer value           \\
            36..39 & ID of the target processor            \\
            512..  & Code segment                          \\
            & Constants segment                   \\
            & Data segment                        \\
            \hline
        \end{tabular}
    \end{table}
}

Notes:

\begin{itemize}
    \item The \St{ASCII} string at the beginning of the executable file contains 15 explicit characters and an implicit `\textbackslash 0' at the end
    \item The code, constants and data segments sizes are denoted in bytes
    \item The code, constants and data segments are loaded into the virtual \St{Karma} computer starting from the first memory cell
    \item The execution of the program starts from the instruction the address of which is specified in the executable file header
    \item The header also specifies the initial stack head address
\end{itemize}
