\section{Architecture}

\St{Karma} is a computer with a Von Neumann architecture with an address space of $2^{20}$ words, each of which takes up 32 bits.

Each command takes up \textit{exactly one} word, 8 high bits of which specify the operation code and the use of the rest 24 bits is command-specific.

The processor has 16 one-word (32 bits each) registers \St{r0}-\St{r15}, as well as an additional \St{flags} register (also one-word).
Their usage is described in \hyperlink{registers}{Table 1}.

\hypertarget{registers}{}
{
    \renewcommand{\arraystretch}{1.4}
    \begin{table}[h!]
        \centering
        \caption{Usage of \St{Karma} processor registers}
        \vspace{2mm}
        \begin{tabular}{| c | c |}
            \hline
            \St{r0}-\St{r12} & Free usage                  \\
            \St{r13}         & Call frame pointer          \\
            \St{r14}         & Stack pointer               \\
            \St{r15}         & Instruction pointer         \\
            \St{flags}       & Comparison operation result \\
            \hline
        \end{tabular}
    \end{table}
}

With respect to the operation code each command may be of one of the following formats:

\begin{itemize}
    \item \Ss{RM} (register-memory):

    \begin{itemize}
        \item 8 bits of operation code
        \item 4 bits of register number (either source or receiver)
        \item 20 bits of memory address (an unsigned number from $0$ to $2^{20} - 1$)
    \end{itemize}

    Example: \St{load r0, 12323}

    \item \Ss{RR} (register-register):

    \begin{itemize}
        \item 8 bits of operation code
        \item 4 bits of receiver register number
        \item 4 bits of source register number
        \item 16 bits of source modifier aka immediate operand (a signed number from $-2^{15}$ to $2^{15}-1$)
    \end{itemize}

    Example: \St{mov r1, r2, -123}

    \item \Ss{RI} (register-immediate value):

    \begin{itemize}
        \item 8 bits of operation code
        \item 4 bits of receiver register number
        \item 20 bits of immediate operand (a signed number from $-2^{19}$ to $2^{19}-1$)
    \end{itemize}

    Example: \St{ori r2, 64}

    \item \Ss{J} (jump):

    \begin{itemize}
        \item 8 bits of operation code
        \item 4 bits ignored
        \item 20 bits of memory address (an unsigned number from $0$ to $2^{20} - 1$)
    \end{itemize}

    Example: \St{calli 3121}

\end{itemize}

For the sake of not overcomplicating matters all arguments of any command are required.
If one does not need the source modifier in a command of the \Ss{RR} format, the immediate operand should be specified as 0.
